U2FsdGVkX19gGCdejPR+0OMwtKJ3/iH33mxYezRSlGZUcA7uJbU32tzAovzEvZ8w
PJsc+zZq52v1rrlICx6rzV2Kzsq4o8k+SnaXat32xJtN/PT0inz9PIdqjxFgojOF
k4l/cDJ2YwySP31h+nzlwji8KHPYBRBu4ivDit6nFTPzbOkSXN7IwqJIV9UajpOa
9oZeL9y++SBlAKfFRwCCOSfC9dKDLtmcZwWqa3ukEcIX/nCj8UqfH2PZZkU2zH2J
wE1bDgqDld+CYKrzkX+kGG/nspN//YWxqLHU9o3Bz7ozQAR3zNJq/sApEf2MbG1V
d5gZgCzYpKoBof30sxsN5gQFy9CAZYLjcd/oWnT9SEB5fb1XTWeI0b5mw6yGaW3B
AHBRsl+8xSmUr4tdU5i6zuANWaKUxIglBewVyq8CN1zyFRJKHg2Sv/gxsfbMpXD1
A2xH5ldVNBhg3pD1yIS1Bfb0tfJzUKZ92MtPUo4VQP50u+AuOu+J4PT4KCBZ4hCJ
zXyrRjzbOEoUXXiMsywGiRiCCQfafAY8vSG3BKD+z6rODsITZ7tYvDni7NWGlTHw
Dy7QriwBXqbln+qpZXQ7WAuwUHQ4HqBmnb1SGB2N24jJy5kwdmprGBBu0YYUaA3o
LMBOiUbuxDmdwP2pMxa0BE/+9Ree5RRsAEUrcNI7FNSkYwaGUD3ZF0+1TkZOb4lW
aYlzGY+h2wZFxIuEcKdoL3MuFzTSIPwtfFH4vEbA28sFcYfJX1/3UM/TGfQr1exd
ysPPvDtlHZThRXYO9soTcvZlOvtegmVJ98lO3TQRESf+cvL5Qj1Q3mn/zGBWUdDF
Ezw0AXwwE+P/slH/13nLbBlWwMNGTeTeKnRFDWUUuatNawDOpIzquEHYcyosO4c4
paXyes9fesMuPkwmGDiSatEhZnzYJQG8au9vQzEWFeNqMznjrH/PGiMQvuQd7f9x
MJnddpcvyq+kF18VA4yAB17PkHrSJA54EW8+PrfgFGbOFN+bCREITXAUlGolnkcF
7LzZxkqOH/WijNv9hXWzmvpbTrXypAEKaImhre2+fi6Itaztg7DWQ+nQRthAeb/O
8D21WcdHcWOkgiCTbLY9RfOjzCdG6yYsBCGJZVFUTWK39VwJMsvPEvcZC0KMxr9B
3UpnOPfMFQcgDTNyESCGsNGAlauL+81E/gBgXm8/tjLZyglCT3Ps4DRBZinLSf8Q
rouwNzDXx/joG6Gdt1vsbaMMFkSONODzLGLgPuP7yCYXgVGBv7zRQ902Tmvy2nFz
Q1odj3qEf4SSQ6hcmTbNf2Si4hMRoU4UrHQ+1Po0D/H/vNuohLglXWDwmh4JG6Dt
Ln6k502dhaN04bgmtbetGCfbp/uxhXZkYHbt0ZrzVuI+UUVmDACAdg+nTIF5MZDb
lgx0IS+/Kk4R3EMAEspfStwzO8nnzbX5e/CigVV6Z5MPj3gFP4eTFpMLg55Mfdji
FoPiaTHL7sz2vc+W1nCuS8d83yz3KdctuPEmWPFsofhrGPZYkEryVRc/C6XPQf+Q
Tl33CTRq5M4rxiW3LHrZOG5LzWytlVN0LNB3lHy3h23gi8La00hZjjcVoyecaY/P
8f/qsadgNN+2idWJLG0tTGCn+VkPGjatkqy4FYEZBHjJudBDAE5K1aQxpk+D8gva
S8dhi9Xarv41C5XXJ1EZzAQznXNSn7QV/uXxZ7Agz9DN6f39/C0Mly0j/epbzIKz
KhW/nb1qHNDTZAMq+b+ikVFxiKuZqahh1M+Egs2A6XEMj8v/uU4WLGJldOLNiifK
zsGQPcypJHrWFboywbWtFH77q8+MTZhUqkN6xl83zt8iv/0nfj5jr0XzN+od2nja
wd2Xa4FTt9XUnMWCyTlTsCP4omubNc3KrM0LmYQ6MTIdaUC/VkytUtM6gEZzdUCJ
fm7CAWePkH9Irnx18HmPdp040RsxEoUK8XcomJWTEyxLgyxROxuUwbFAc7IWo4sP
naU0p6g81vgq4r5EPu8jNOj7PRpIn+5k4TBwP7xk+mEuwgV8ozQhT5j9goPYceLk
/BtSX/y0z7uJhrtpDAVQXTfS9nxx6iEXEe/Z9boLMqREWzSGbci1r9SLVflcai0N
PNhBRbAf1hfWq36ktMRGTZdVdyyfkdZraTa+DBtYP6r4e1pjRlemJg9cazX7NGHg
E+Qtzlk8HPGltu5tpdAJtgTl9+FCBusvD4lfBd0hgqFHfQP/eoyQg/PI2k04z2r2
MBEM3TiI0zPFikTCstnB12vXiDmiOAzXLPF3bSg0QtmBdop2bnXbFUflqu5ZiXx8
LQEI+XTqogOtb3XADXY8ijydiGDDwxrNUDvAk7B2PyjvYmv1R3uEAmVJOlVDYe0/
39mTbWpowahtzrBOR5QZ2ZRYWMvE/QVhjWGrRqkeDodpzlpYrEtqlEb50fkWrSAj
CnSLgtWdFFQY4AqP/lDpi3D4PhWvhKsdkzUILeI4UkMohTCMrURQLgzhVLZwt5IJ
AVEhdjSIij51swV1RkTyBSK+Rtz1QsboQInmzgWHdvTabhZaUTLNruWh2NVNeSJI
8AirtzPC/2Ei0SBndQlRkKfz0baYZ0088XmuUZOCEtvEM67c0dtm9JQlpPYstg0H
T0W+5MgepdZ3O03fKa5w9PrzIHzPF0oikg4ivl/bq1gdIP5G0xsqj/syKYbf04bx
hZ2up4ogg7HEF+4D1nCxyiIY2F7rBb+Rjd/QBh0sqmyzwUbr8bTZIqlEkLGvDioz
jh++jKWzZdHK22E6hax4ERb1oGv8m6rW90b39wUoEdhrfDqZ9EzbSEnCM1nGpNsq
Zyn4aaMZH+D7AEhAGtabeFnqwALxvhnTQEOG/d9BS+tNG9gzecxud2GLTi2vJGoW
5Jn1X+0rCXwQ4qZmxL1FLIpWRhk0kmaLYaRgt+iGGCvxAC6xrrffsmwFzufaMAIA
rygSLpjBQQpcY3Lr1SEZwM/UGlsPJu4MjYGQxddriDgYLn0hykPm/yAhTW7Snoie
tshuGCjfFw5/D5hwVlotzQIfuFhVunHMBIfx8o5GWyXn+lX/yjghXzREuhWCk5jX
iKN5/G1kIii1kXrVCodr/rHFtLjxD6Y/Ath69+B/cE68rqtdD8poJmddp3vqPFZi
4yzI/6tHdyBvoUEDHXeK0gciu+KDwJTSRqQdVNK+xxA2qyFJwRlLXd8g9A1+mnr8
CzEyusNtk95okaVbX1rOus6TIRxQMhleY1BdagnR9/GNBgGronBsm+n28xeJ40xA
43ov1xlJ4oo0pSMrjCgcynFxxlo5qybgHcFCk/Ppyk/dxNHNeWb1SdfMh7lvl2Mq
QZtVqFssm9iV0S71bS3AqJt7CxISo2MKrgHywUVbJrgU8EO6pVsXW4OBpPpvqOSH
6BRvjIpBasO/sLFllulW52FumUxMtyZiK6QS2FsBA4lzuSnbBZpr14JAqUx8z2Cl
wpkFLmkKzSMMUvq6BFcAeyzXf5MjhLgNgSLbXtEozxi/PyycWgqR45HzK7fmUKiQ
vCmsGFLr2TdZsolCDODpBvvq7Uejvv6uIFMHu1HzonC4Xm9k34FIcsfaaU43NN4t
4xsRFdFyJFPH6j8KJeGQNyB+vMX6mhM/M/Wg4U+nKPMDknRWgMdO0tr/b3aZcf9w
peDFK8lLGtmkwAARPa3Bw1aoLjpHoU4S/ltL8h/O9TTLVQCovXpy+uL0wg21lBz1
WsqYLn111bHTuizX1WBK6uOeN+Bc/pltnVKBUnSolAns1jmFgQ7M5WWENgbkR3ah
n2KPNSvuanWMbTil9XuSZVI/wumF/hehiPj6l3bapQJxR2vf31BBlLfbrZoDqOP6
1iXfQ2X686nyTUlphNT8iwwJpCZ5jpvchkBEb6w9STjCJeMbIolATDjVOXF//uLD
c1OD7i8HmOaFUeXMHHbV+VC4rhii+UfiXuXnKIvG6tpmHFipRl9HOSNZwIbTIohP
F694bFWJjXNBZc0hjnVJy+I09//R9M11M8XJyq8BdyMt8o6O+c/jbdVNNG3EfDWc
Pv1KnOhxhxddtKviqLwbsC/PVvoAKzx9qSxb+gpqVS30J/LgvJ4TErJ4ByEPZfSV
JH4juy8nLxxi1M6C6trdfDuVjkGOhFI0sURledIDa0eGVP9Um/RwbsNbbnnjg71D
0fCcbEiyrbRsZrzJhv9JyRTMDEoSJsmfdH+jLjS3L08jAtsPHvMjX8JVbdwU/49f
zzadZhS4jhtK3g3k/CEQXgtc81ZQN0566CTaVZXrLC0pFRTx56ooblXljeniTPme
gW9U+iRmu2/+8Mf+ok5jvxlT8GlvpHwuUtunGKrqkaJ9DbFogsUAeWVes/JQPzlt
6VikVu9okyAQrcnpgst4GC7k83sgpAdiaG8UEDoIkpLnBI++/mjh70Wb7+peS4nC
MZMCgwRPGcQckgUWuM0zWkRH0mTB+ejq+JysdRFXXmp67GJvI8WQRKlGp/oP0gSn
buyNjWKfdxFrtCt1g4leVo7Of0gAVqBoywwIWi68neLsfK/Wpn6VpaL76Uk9IdUR
AQwDb4ennQCyZ9Lfld2gGBe/joOyigLy8hhQIdA+EVIZSvd/1ZCOpQ3DRPW+B4Zr
UgJcxuYpwyhwdJ5Id4viJzjjYxtt6dCzyiI+ZLk8SEHblhzDGgR8yai7p4EYUZa+
xmFfmXqevFb9jEnecvlaR9ju0deBc9W3XDyb4WDBpiRH5sgfG7OiOpN6LuaFlXnD
C4JUBzFfES2GOYr0CT3/wo4VbH8P+T+JSQBzagoGREoUebOiL8Bf+rYmCeRogvo0
qm/jVSCJ7r7esiVmzRWPwPhg0BQ7KJfSjM5WpeCC++zSFfMx8YcR/PnJFoaxPkpS
zAS5NSQl0DDU594EbvLYoVX5PimhkHSDJzkfuyCpxIwxgsA6hAr3X6wHJ0m4keOI
7O5AK8b/XPlsAaxMxrg2FpyDMZM42qVhzQsUSQPI0c840p19+Mxg1l7qEb3HVA9y
KB67z50vkoL40Kd2z/Z23yXlJZR8cI5y8n5Y43K/ygTmOnu7bQx3mPaYSUEQE077
hKovMG/ZC6/t4no4EfUGgnZ6RfJMFAkVFQ9wJ3+4Uvo+5n9/Ib4ocKtB1aWGnXMO
o8qVLyXkqaHfDcFt5zh+ZMsFzLyNbMU75jG542/BHifIWUkOS3ZtvN4To26PwXU8
IJ7/O1Simk9BE+/1TrArQO9uni7S8OWDrCc1KmI6ZKhgbjO39qUxEdjjlD9JYAvR
bujDgsEQUgSucJJ+iNWqoE5C/uGvu0UPzxyq5pZzPUbY17OiaH2iqfnOQzdYjEXu
sG+iKKPXRTY8H5EsAlcF2zso/4Bnd2ZmOyLpt49ZQhSO5Nby352wKtEY6uD7aQn2
CKv+fA9WZeKh5jFjxXteqLcEjLli++4JVD7wmbX8k6HdkAnMxdo6BCHeJsC7XVs3
8/aD2GJPnt5z1b6fIsV+pbBfsCOVUKURYU+m9ofPW0aF70+i6se/FnelSX0z/1dz
1eGJAQyk+yK//6DdZD1j5PwU0yVz9xpUHIpJvMvX7tEsHxjsHddAV6jzCi5Elz+2
EpGDqp+94nnBrjY5S7kiTW+ZPGSm27McUIIXv6j0vSUbYvvl8rFpjakYUVvZ+zoT
P7H+aNM/C9h6AVsI7BtZHK+s5MfWJxz1Cp83WELHxCqsVd8H9I5PukH30woWUUFA
gSPCxPzuVBssv9Cz2gO7qbLTkemf4NiopFDUcjgsnccYLOICDXXj2Td8JOcQRufl
KqamaENUJD/KTztZts6C4/BgAebej+CANpiG5q5F6wL93eflciOzPVnhmztGh8Y1
pnTcjHX75/MYuQkCbQzaaUmBL9mKOsw+hR+IITmSJ1eREfZrY2/mDDIDOooLoQ4o
OFycHYusr7pxiZ9iMDiWS9ldajO4vfoUmX+NTfNfzlIYYUh1mHU6frtiAarcfD8V
XnSftt8VW7I4ZOAT7L1gyg+aSoEUXYvs9OMaSAFo8V8C7C33NiRS04es5EzQudxj
/ZOAeWSqaYF2cGUr4ahJ4zMkTSQBUQhnMaeaZno1y8jNmkNcnPRs55444CbCBqvP
32+9ZDBgeL/VcChm0MFMA+/9jmxChC9tg3q+l/SsN23/WpGOF7Nfjiecozz0l7U9
Am+Ic9LWzUmRnXBDaYpX8AqHHx3mcBuA1DsEZHh0FIGw4ZoDNMan1697ZTr3T1Lp
axtCn+YPRks1+1zYj0jf25bWqRlDzql4Zc1XPa0ahQ8gylP0cze3WZOl2Jb8olJ9
+9sggvQJxLzTH7Nps/9V89qHeU4rnYapSytYEf2QT9jVqZLFYw9QaQrKau73T9wd
+eRb/XrrrDDm+znRhbebrvWn5dfgKNSRdloz0M+SI2jCTQaMAa730hm64kYiiNk4
mNxgacbAm4wwl/MkL3wnzAaqldHg1u25QHiHkxxuOvqPd4pdgKbEIXGsZ41uBFGg
jtlwOkTzyFh32K8PO3AL6W9himU8A1RdN+58nLeiMC4lA7arfa+aM383Y8Ivunxt
t6cj3GbdPQIGciSqd78f5oImVln8/bijjCmOteQnRvW5C2KQGeil4JOviok5ePcd
/kA0V4+vyg8ZEJzx/0vcRrepBPVYs6+KD6m2KqaIlgOvPem1bQynNN58WWCOZNEs
5MykRgSfV5g+B9lIVvsXkXXHgliZKFq4aMeoswlNNGQIKt9vpXugl+lmLPXkSA5/
WN0/zpeeirZQT1IU4jONOr5el7npcCcF/Bhr1ewR2a4X44/4G2rPJnVZRHLPYO5D
2p7XgsXrXrqGrd6PwWwh60vj52dpbhA/rHwy7abblv74VXepdne2ixHX5cVf+4sw
ZZAx5zHnR3Fz7TDVmw7iUz5XOM1Dm5E1axzYmv4Sl9kjMWXScmbeRhDId9H3xxWO
AayANSG7dgQCFeojbOxR1idbI5ni7XZ9y+TPiQ3mgS8fWjiOHdvIFU4PFb8clK2A
YdFA3gBhGnE1wxc1PU8xURnjg80OiXJQ3TSeJTUG3ObK3SfCERIlxGD+es/ttoaL
kZTre+/zAMv/Z1OfujPAl8itWWpDPROaqO2CcUu8oEroCpVWRFfH7podp0hvrQID
PoJcr0Gk+bIhNCHBnMJTiWukoAYJsuXnCUxtyqT1ZMaeWhNKB9iTxFJvQgym0gEh
nhrj6RFjmB5GfsCaYZiYUWA9NNwSHL4/0d1rNaBfUxQeTc6GCGE0nKEsEarPe73/
gTE+Y5rCYteGzBaRrAN7z9jgMjb8Hqo4y4bGOEoZ6aqvXe1a8rHLmb6juCtPkIr7
pW8+aIuCt0KlQWVg7fVDYtt5ikWW4eNa+rjICcvYCrCzkt+ijLA9GcEw4EP26F40
Rm5PUjjqhki5nRsmaHRTA8uHVXpP0Q557fdEGqcWzPmXR36PlorzG7lgbR8RyYP5
3eWBtUp3DA7oRdll800d4woIKRhMfmvcg4ic5AsHBBwobqHfDJuiN/M8Vgui0FYY
h5xIJ1+y8FJbrxn8ZXqdH3u5Nib56PYulWCSWcTD/+ksuglVZY3lYFBjYP+lBjwi
iJHujj490dcUzULKCBXVLqMYO0fOaWXqhbKsSNlC0ycdgkT7kHyvO1Rx0xa4HDSe
OHD/3Y224DijxojfE9Uh+1DAB/7avB4Ynir4vLtb3BCsAiVVDgXkezKTcic17bS0
5sO//Kuc8Ohe3QdPAHwjPHz9WQRGjDIDFpfiSlhASlvfEqSX1O/FrlwAGxe7pCDH
C0HbaUUuiDOCZXNdcRWaQ/RfdAqKHsnR6ZQys1EG8U10ZlSwPso1QtGtdFyyIVCS
S/hXCcoukb1cJCXiRrnNciZtpeS0GXjuTopT2rX0jYnfAFnKIsNc0t7rmj1jqTW0
SvrTiKh8dOak2O3vRwegsvGtxDLtJkG6w1s0GuEcVDc4VN0RGeL2bH9auoZeksaK
SNYIn3C5a0HdQ45QaRVnoWHfB/5DQvYlTFwmHjE51rpd7mnbu7MELl5X+W2yUuqV
FW4STP06q+UcdvEIodEmSAB13bmWGe9dJi7rQF8NdoK4y8vRH0LKwKPEbmuu5wvd
sU3zkbFil0/eN9ueua4bJX3lgl1AvjOwBL4WL56RevXoSmonc2C451jSVeHBZOAO
4IuXzzpyH7xnKfkFEx7q1Nb5l73HhPS6xNwQ165iDei8ucE9xLErmcvi1uVXxHUw
7SDTuhp9kNzDYlhHvVyv9aYF6Vuz3FKVvJVsGquL1Vd7SFXj5N8WI+/1/rnNXOo3
jNIZrHpkg5DshEvizxxy8hGpyFdE1EJ44IrYZWXyO4PmFq0+BmEQZ7K9OsJh7tfq
HV4jUwCtui1F/ID35kRcn447Sc4gCXKjnnA4xM3GboX3A1et6S7KM9RiBTfy8Yvz
0E2WKg4ct6mkdeC9bnQCFN/eF0SysdX0KkMoykFsJK6KTXsMvDdFOASCOq+ZjL+3
NCIoLinCQtYxTo/KraUsYMmBDcpBzn6T4Drf41XzAmy7+Q/QBGpCBJCCkOidpTOR
eAW8QeaZwIW3ysRA7UoHaUQloa31H5gslbiCSyxZCZbzZzCORtcPjTkMcFJFq/fQ
2Mth/QrPQwivroeCD2LfYMUv3LbKuUitITKm/gCcTFD20oio0dtF09O+zxqYCDXY
AbM0dCVAHoTBHqp/MJsDaOBe67+uPrZ0VRbo31J9z7Tldy0w3JjEwA3dWfYugT2o
bk4PaMuspV8HyNrjfWlxFxQW3UV5eMEGW7pQ3QHaNywxzY+cuMfCikoWnqQKybxO
gkEkjXCqY0B7korhYeMv2rDfx951kYoReFcO2f59+9tqMcl1beKP6WtBTRA6jLv4
tnBQNBiJk8qtWUDNBlJo3YhaXJnLq3Si+CTGJ50bsNBjfeItnnIXquc/vXAlniRf
v4diIoojA9ih025fWMXRfX9zp+X3ozbrjGmKs77EI7YmJdXqwCFvf3TlS8zhLOYx
DddxD3xGc9w9t13vPjQmsLVd0jYYqgiUyRoOxGRDOB2K3dUBoEMpfmXjaGqydSEo
TCVtHFUMl3SCABGfi8Pwwr3AYEzW5+Ozd4YNlH/9p84Koe/sn1DIL227iCEu7zbx
VILinKBw+iUNyfz4HXa7tEL4MJ0kT5DVchGEmp30VMkU7pQoZpLatFrKVigOPAUV
iYJVTM5WlYPOofNqgn3jOzQ8DP1O4OchUZzcHM2rE47+aBjfXPESYQ08LkG1GAR7
DDRxH+mCpJCgyO4rXkIoYc06yNk5Sk1avgtQy9HgFTirmhR5IZUGophL/8VhgriR
IJuA2L9Ya9Ci1h0/njXHwnaly3Jb3MVGOFwjedPbNw+Enw36VzW/RDd/qjCA0Z7a
puIwIpxGp/gfkQjAYAJQXkfZOWJehY7fQ7CGoj1GPWS+yfrqTxZ+yntcjBywUZa5
wDjmcs70M7lAhuOeM4Hm7LrfhguFxouto60+p1XWB1/tYu3ujIt9BphLGFpf2s3d
jWUqf/Tsg3ZVrGh86rp8lRPM5Ya06e96lLZT+NiC+WBSbxV4PEesyrOuQsJ0us+F
5k5BFbugDOzXNifm2FkwnAC4aNpBDZtPM7DLvQiD3ZvRr7B7ElPiSraPpQzdr/gs
uQAva8M+KgPNu0kmV8lfKNXOIb0vsIN7hCCAtJuarxCDd4zDDtU0Q7sKGcmAtZP8
YpblBBZzSoMcTp+f+3w79jq2/vJAxF8INJB/+rauWhYKNUzaEuxCWgnkLcArht1/
ByU0zXsQADCJ69PF/TghiH1x3kdMk9VUESsRzBBmG1UCSDBiRqrTfua+QYmO5Dvr
B6rpHYNGYuCrpNAmPBHzGETrUSK7vqVRijJitBGbBbxBNRhlQTmdYBucKzPR4LMK
74y6y45z5rvEyo55epU7IAeqrCnbxNv0pT/4Vdif88ROI9inIFcHeVfzwqDQU7pN
v1JAfo28eVs358JDtZH6lkiLP4BT22DvjfKWPVV2+J1e3vtDoNQj2mGlmW0RRBhQ
HPbs5IgjZuUP7e4OmlitjPWL7so3zZEOGnaE7zlqKokpRO5cVgUvIleXpfhWwkYu
W3W7uugyOLgVwGpIIw6CCudpLzdp6aT28mS1rgDIaA7n+l03j+d6OKivOjfhI4my
PwsItsag1HdRqPPhPxg5Jv5tGyiNzAIajyUdVew3XTN07AlMcRkMBM9OKgiKz2FM
ekkiNMcW3uymCpgu4syTBfCvoyE5npEB/YD5z+1BbixEwKeTWsViyJpOds2aBDl0
nSLuyk5FaBSc3IbXdxjwVV9+cHfB8L+mKD/kcVA+JURgv8cDTdFNGTAMSXcBTZoj
nVIqYHiGN0LhuFI5nMCsszz6TC/j3Jn4Y3usHhsbLpgGPQxJJQqQ2vWXcw4K+NQC
Fj722wbayw7DmjqyCQGkw5qlW7m+tNC5goksF4ge/KZEo4E5SmI6dBZlp9WW/uU0
wIlhy7BZo7k8rvCcRKoG+Pd/pF8ha0+hqaGh3mMsT8ck6SEZyLFT7AdTsgoccBe6
TKzi6ULBXQrazr7kIHoG0PdpvzNydgcWeY02TaFl+Ox+9T1kLTgAc1wvmriTYwHM
49lpU1Zl45SoXbFJ2ykRcnZiAmopqi84wyPU0VmNm8ezvcssd476JZOiwTihDffv
REadDHtrht+yNh3lDDgpIbKNTApo3RUmRc/bH4knMyBf7LFRvH85ed/NbXgJX/LZ
sbk43RpQxluvCPH5CsoV2LwxJ2LKZ1+SUrPnwIC9W++G6Jb7xAfRdj/63ONqP/+L
ABKazm0ScjGl6HhrUEv1JxxKhA7r1ySdYvOlrTI+lYxoaDViD9it4zwqnYPRhlk0
nlbBPJ2sId1vhllV+ANGBzcTBs0ke+t33amf/wRkNYos1Blep4nyzrGLp5yJkplZ
a/lC7Pa5xUgC42VtXh/zdan6ZNtNNZvEE92uwMnpFJsGscuD8haT86ptDRmNrcYV
R+xhg7sljPB1/lZI9V+dwz7OBI8/PiKg+l3TITqe3Nm3DRVlqeiBo0bjrEyMHilY
kcw0dQi80N+eMEpWrSaj4tkyRnHsD31IPbpOFT3KP0h39ZTyW+v/yAV2V1GaTyWy
xWWFOxDIUQBKBGadU0MHeFPq8gM/V9XRQnISWGyBXpNXJIoJSvGNICPyi5qbm1N3
+m2D66prnVBFfh6Lrhm37lR+JHQFgVPASDvv8HDWdRNa8Bhet0nhKXsubBRwFJEa
q/GVhukA0Nr83cX8u0PF4dIGBI680aVSq4gRhVkT4Xb+2m0ewPmUQuwl/Q7WIkFe
yLXjGf+xisI8G6hsCLBA2Zk1Skvy4k7an3U2GNk71FzLAYi/4+Rnresd+LdGFJHq
BBOyE5mUiTvj6H+5CDTvj8mymwX9mXuqXVfXLmMfj7e17u/M69uMNAGORX056GPY
hCSjFO5JkT33ZwvxMQo6JR/nd1C8HkOWeIzwTCG9nYMZRTG4ZdqOoTXaye8LjeVh
ycJ/qSOP5S+0Qo4LSdIY4KWhAYPe8d+t+YPt+MtqHZEkycX4tVHtWz34/d/2mBg2
AMtVFE4mhA7HgP876sW8rsoC9QJw/Q+6YrwnOBWMN14+vRwg61Kz6HU32Y76m9aw
25XtfwGTa+TLQ9CmjTa2JvUwadu7o5zWwCsX1WgxqfJb8sTfJegbyW5tp7xNKBGR
P1RY3Qa2Du/6ovjukOOrWCwENIok+zwVczLp64JwNQHBiDpkv8GMyBgZpM0N7Caq
G2pnCCDVgOq5FjmWKRPpEnIeiG6Bxvo76kIyJD9s74DGBsW+j7JokXgQenNl37IA
2vBMzBGWRcdT7lT6oGk4C9AWPuXptBsES/AoaqBrTlbPB31IMAbLgvxZ9Hj+xupA
NsCpM9nRlFVHS70GUtHLyXf74e1KVEKeqQqbUE4opri6g/UKPZkUgFsXx4TQzlGo
F6r34ijATTCgHCttEQEo9m7xmlvr1QngJ50+TE4HsfPIy0HY7CXODoZWLvWd5oLZ
Khw5ZOouu86TUf8fXkz1FehqH/nDvb2cQxPiPaQYlWOMr2pqZDDwMZKOKy6J/Z9m
hv8Yu45v7WLPc2q1TjDzYyNb+8GXvrWc4fFDq7E5QATfiKpt8pK+KqNwhSL0z3nM
3OB5a5QDnWLfqXlfvE74Zw765QNaYvMV1m4mDWBVbv28PIHAQMswozADhwjwFIJA
ovObj8SZjcyJ2kCw+ap0HMbuaZ+apze2gOmo0100z0SRiBchYe9/JbXM8lPzzo5u
LJpUQ/R7zyyZHRo1g4AiukRELIFJDDtJde9LN44W8gQCQNTG9zWHgafUGFC/7d94
GqJkeq1RrGANx3lZn5KrkEH8rcnqm4URpfIFfHM8GJGV0RG2ovnhbSJ6fpWlO159
BzhJBJQofmOdR1HM09ORMV35i92tEjdD7OTuaOUs6nqFZVy/jy5gYMSRuJ0B1/c4
cYAKrGs0mtfyoSr3l6RwhFA/o/bTZVHtQFMXnr2Pbi+LmWhe9aAWrZC+6QOWDKXB
A97UwatuHrgOZdxkqYbpiewOMhu72pkGVDdqFFA6mx7cew7RxM3ZhOd/y/GxCQvs
R8A+B/RE27JcNWv+p1uTP2Nlpzrh6EWKZ2QL8FWrJIlkK0bxyZJXka/XgZwHotmX
iwojGIp1ccMI1P9wTEtR/t9KCMIENdhvSQkUh3I3s1r0RIOvxH+poq3YiCNtZGly
Jtbzgez/dvDkretmgj2QXGwgbZ1mVHLWO7jcVtEWQy0U2qsXfXt5s8EzDcZ0JOBU
7A/wECjex+SRKi6KUiC+MdU5xFuO57+OV/JryRMCTXY4c6hx0zRnXThv8OQ+jog0
X972n2z0129GOZ7ouqaQ6hBcydNhr+CLw1Mg1ThiEPBvbOaJtRJKjRu9OkW43AqU
DfJNMRfPAgsxJgWILyFO5W/9Pt/TueV/zG4D0aGAxdRyWjO+Bxb8yeBeGWap70mQ
nxCn7wjaY7CIbzKRPIzAv9F/ixi2dVgiKxHV4qkEN12btSgBN7yr4AeAVex7ZAfO
qw5/JCMBB/qiRCMEpTKCkOflC77If4SgUKv8ghUDWFCQfQPDvrldCJKvMOD3crAz
Wuy799CDWMux5nrRvff1ovJBeIYDZ4gc3wN/Ul2LN/hRBi1xODd9A/HFo4fuIwtf
kJ1HLzMHuxPDABtCsnnsbjz8Ml2LigAMXAnthohXXrrMBe+MgmNDSQ+qviw+y1BY
KnbQPA2EuK+cSC1BcKtQjVWNqIhRi4rvKf+l1V3ZJmqvDkXqDMC2h4h95xh0PhYy
sI1u4iX7qpCaVi33ngsiSlg3Y7bpCTy3XM4TosKe0x4FlpD73XnTxdqNh1tT4Zgd
OVGx5zQ7f5gOXJ5K8ti3h2yX/qWii7VItg2z2WdQX4U9pORjd7lmOm8JhgTT4pQR
oh/2xp74riqRdXjjlIX+u9ukQp7uWamcKWcEE8WBuhwGJKiz7ceFsv72+t5W3DtN
mx76OGFkziR6d5yXsf/d95FWqe9KPBzeHl8noB+slqcTJbSBn2xQw33+tC5cxeX/
czubrY55VjYQgsvSb66HeG21mWSnzDZLRuck0bS6iTNt9rzFHx+orfIG8qrFmWO/
LUQheDKdokmK4AVh9Ze3F1gwlZ/NvlIZTMg8bM/Ao4cN/LLtYygXRq69HECSGt5R
IiFE4pHWCuXE4X9cVkPs1wHyOJlM8zdY3//jn3Ecr5g+b4kflbQUAUaRIEkVoJ26
N2Nw+HPikBCpCW3AC4tnOhEsFtMe2/eUGRWIerwkmN0bxDA9q38GgkALCTvtW8JB
TtbT1/GHgn0j1viVsh2irThQF+SA5Kqh0XRn/+q03axNPVVtplxHW5MAK9EIPyFx
hQx7sIVq2YV0D+4vM+YgvNT5bfIaOWn/T5FLJ0aRilIO9zTyyoLRMbwsOL+TSWhK
jdWEin+QvYKnzASrrTnaDO1pPtzBvYkUy9Hvi8yqW/3Y/WUIYWE9IgvEngmuRvsd
+R4BU/dfsn6lc7oRoQjBqxQnIIMXs104K06i21ayCNj1SUlLA6RVJT5JhEuT+B3u
vDrvsKxrXn+mF8Lzuc9nkeCxbJZQ9EHrQeZISBuHquomr9JZb5WzKvkV84Hp/fdA
9DzD4yWb8B0B30l7lwhgsVqNhmBX4y0kKxDIcskkd7fn61+wzgtWn8M/NQr1iyBy
Nj4XRXH/9A3dZey3sIbd7mX7e+UcdSnH1m9p+QEXCPt64wfzg/JZ6aiAJA2obf+/
G8q4KE059JF2XWjJ7vqEJ1l2E+FVVnJiJ6LNlHgtuTmt9K7XGJ8qe4NbAiory2Vd
y30jiuJB/e4iT0mtkFhcyYKTRgN8Se0wDG1U42jhZCdRdLHmf7Om62LMSZkShEjB
J2p7biLwzJR/PFXxgQ02LGeyRsqF62Gt6seof+06UHf+WIkH1FcuRBAt3ACVCUHV
FfuVMnfkZYAmb3aj7wDYgKXFuHgk4px1D1zZuk1lUpmCJ1k+G9xXIT6lfjgy3fq/
pjrBMR8X0Bo3w3jTst6L7OavuS2BdVnEJgW18L+2orNrLTUsV9WCn/mQbaLKDj6o
aEDuz2k4j7uobpSO/eYljeXReiBzcWEriaXAeEk1bIScJdH2OhQidU4JY+yGf4pd
BmNpUlzo6QW8D+SXpqT2ubC4tarqlJ7PkGWeImzHBu55FqJFsThnwb/YHnbqwAfI
2OyW5rudEMXgWIaPuLuSKOgvCvB8XrZ4/tZsY1s7AjVhpcrMMw1JVV72rLFj5Ktg
tjURhHg8OtSk+cF98lFREz3lco1YZw40aS8P3FsDjax0aWWIfPNkfOTH13s+6IlH
3M6jzXWoTmWjrmfdm/wzDgOxFaeWK4Ak2K7HLTNscZ47ztgQwjMvOXuD9PbfrbtX
LbBP2LuXIcMGKzeGabSEJzepros0WyesKWxL9KlbRV/b3bltEPznhnfMzwo96NQo
myPwwdVJIuFu4iOkVDLGpRJTPSjl/Z3dLQhqQGbeIFOzM4icxoEO2ygbMi+cJh6H
ASh+W/wVBNcrxGmmT2bbk60mLv5fyVmdVp3ega7pk8nI0fk5ZE7WHwbs/+Awlj8t
R3JHcs+Heqx4QsoZu4+vT1Srj5q8K3R8ZXuYeCGyuMKMZeNeSqXw+706fKNbXYUn
Al+VCYKapfiO0q2CwKyVfQL+1DDVzU3VZs68AVxFCne+mZcDG987lmTwyXCR/s5w
Z8kYXeKd9AH+9fbh4HJXP/a6AN6eILbZITh1SZkj8V0Fi5mvxOBVzAseOqM3MLYv
Fgltpyp3RaJXNSJDM13sicu0hiL8R0vagtyxeLVqmzP7vKt/l4ndjPZYbOsnXiWp
Ee9fNMFaiOYlTSMYwOiv/nEeL9zAGsnJCXDshWvmv9v0QCApeAtZRAd5JiCCV3n8
HxA10YylP69vOl2nmRhuT0cIEc+JPj+PL8XOpMhC1IirDpNT48sJESwOpJrSByuk
T4ZScCufjS68o05JHuK4IqTHmkVgJhFBAeR/d1+IzRF8HGcRn69TsttpdAVIpPx6
ClYBZwzNnGnro6s9BeaBB82MaVIjITtqXRzBpod8cdYUj461ksmvv6eTX1XQ1T3l
8uq46yek2oi5pTnKuy7Ea5dafLl6uFT5zV9JfyXal2fLB6RF33sWSDIkoVp7Nddy
ZWcD1r7whjwv/pcm2WbmKADfubTiCtv8QrEGqNj3PE+nDgnzQQWqEwKSrf9lzXXY
Z5psiR9FqIuBbMTDM71lGgRMcLP1z3So5So3eZ/Rv9spg0kxcvHrgeZMy/IXt0eP
fBwA8mKeqs9gYnhDI3oWgLpmPuNlbIqAz2Ce0X5WHGsA2gjJ4In+zTzBQ7FxRPCY
mHoJPXxAgc4X8D9wq0eeaLJI0uPSe0jGyynCAq4XmHbDHgO5ON5ah4k16Z8KGJWP
1LMPuZCGVTtQEXJMzrVoPlWfc1iZ1hsgBhmC6CtIVTGo+Vc2h/On46CxqcQUcOF8
OoeB6n5piJv6pOZXHF3Vay4GxNUgzlL/DsSaKqEFWotpmIQQUARUQPpTtnn6Jlgv
2lKnCziTW1chfl+OAjBVM3LmXwDqC3pxjMyziDn7gPnkx2RMb51dLmKF30mzGU+Y
w0Dh314r0fC68npHgR0zCJf72+IFBDTarldrCecLNYmeiaa9LgeTkdSfqdL+VS4N
75vLXB0lSAas9X+KeVIJ1eavX15qjP3j5XW5Vzu7gjRh8xvaARvHTzPzg2vhmsbu
ZoZCAOOIF5p6tDn1U8/gBQCOsF2PTHbeavAp/ZSA5kJyEiOaWcfb9TIFO2BNNvzQ
Odywbof0RWtF7hq+wfZIfgORmnsuPDJ3IqMcc5KLInCXVMve+Fy4lS92WgKZVx6S
I5UsFmoxdpeCcvH1+m6DNRcra5tyT3fsOe7sG3FD6NgX0FalkZwmKFsApCs6qzrg
hJznCnYoVaxfZQDRZ94w6QRochQMF+ONaQ8dD3u2iSg6e87AxygEqgmbJ7W3/NHZ
5hDXUdB2Zul3E/MIvVvBX73QYi3QF3cXHtKPujBcYldcua6Mhalv0n6XioqF5vU4
8NT5Gk/HGUcoUlObAbBpLYclKleOnzQWcn/L84Lykp4YhlcooLxQmgJnrJ3G2WAP
eQ/b6zvgBBWvNHs73hHsCYnHBNoKBLuNVayAastlf2XlaSOzjhzPkXkZiuDOol48
jvEtAgW0YXd0wURKCDZEeyXHyRUqOYrG0dEX6kfLtrRTO8OOHQPjTQMZI3KlX8o3
0/UfcG6NBQ2YPNvyVAQdRKg2jZwOOg/azD8kzAjXiw7iAV4FN5JOBIKBIG0Ns3zE
7ONeaZs8NOS8hvvPmcpHiI4yx1rjSfoaCxix4meFzOk0w1APvRds4jnjezpKIw7E
nKdTs+tZK+NbLHAmj9HTXuPW7UQTAzUIbV7jWvfwaJpRZaMcxvYLj/kGMiD5Dqsl
6v7UzKxPC79QsNikDd/yFxQZZNKK2owjJrwjQRh64vjnv6mjuuYACvXJxgh8F5yi
+4TySCMw8CGj/4Y+YjLHtYmsNvQ6EyoWIi1snjnilgy4O0zv/Bx8BUlk7YHIC2YT
T427l4SpH5wNMYLBxhJz0Xy1zTtDxy0BLQzPAt19b3AreFLp+K4TrJHvHSC6gw8v
Qn7hs0ahyxufjemJsJsvCBj1V5X8BP2fUkmejGsKD2Vx1Lp+M+Ex4L/IIt9TEa4L
Uw6OkATw8/Sl75i70jANQJiAwR5s8tpDOpQa9yj6kwvcUpRGaZK14+6lzPyL+VOe
xfYsKDdXhGUgdjtVlTn4eI0f5GxM10Q2iRFunfgsNTDOaco9QtFDUYIS2dFugit7
MnIhZekalcPCizeKjk5nms/W006ihe09r+Rkwciby4Be1jI6OsvQRYIWD2Wy65wB
Nn8mCEohDWhIds31YWXuobsFqmOs/RcCbF8Ec+Dsb3+EPN14W0lSf/WuZZ7D7d+i
Jp4wS6+ULBT9IAH8ahG/XhmnLSDJp1WgfboDB+klds8JCGlwLzILO6nEoz6hk4hz
90TMoaKBbR3LWLpxrn3rIZq804eYYdP3E7jq+u++mpFallEfc2nLIH/LLOdmhWts
N7m5kAzhmf8ZJ1+shUEsHrjqdgbxyOgEMPIkpUnqrbL+eaymkTcFCUTtdp5pL0TR
72IpxlTXxYpepWfSB1+wFyVzOLdlCjGrsxp7POr+gfaoRi6WQkTfdhMkzwoEzlgV
JnR6789zaKI2xUs5TozoMfMbIfBB+uUWwPaFOlqMlG96Rue3U0eqZSBMQCs6LWlx
41e9hbAGj3yMlWktX3RJt35RKXZgIUUZfVYSWPTTF5knqz+L7L0axj6FQryKCink
SN4zKdfb9+zEbfOuxzhx8HbaVv1k1MIHXZyYTlWcKejM5pTs4Qo4xpJxnLYnlWcs
U4RM19QsSloJJ+9jHb0EcN8SFqMcUzjKQg1J7gZSoxHLghSZIrvI+N6LSiyE4AZe
6toY2nYfJiczkhk7SZ46ap1ye6Rikd1xAAjZHIv290DDE0XRSoajXJmtqYm0Ftnd
kTjIStj2IrcNeAnIsiBDNEDHX6fq8hlWTNSVbU8bTWQUdYPzSgwZIN0HFOlHGHRR
69WIosXzMfzxYo76Iuv4Fj0c2SPJJNC569wR/b1zNbqiX58ZxQBYDyoZcjR1wUsm
KlRkU9FDI2Cp66YC6msrJAaPMAlbGrYHSnb31MloBaA4myCSB++w0S1c7P5aWSCz
JatK/CCm1aZII5DCh3ti/MZseGDZCmK5uc/fc+yusSH1hzp2o2CBsz7nBf+59VNF
5qKOLkHhtyuLfwbJSWpucxNI3+ipqthwe5xgQspVk8BSrG6Fl0BoXswbKbKNbjHR
guhB4oKc7UHrCA/XvPEQ4xODuySEF0/gSIsEd9Y+rlvY1n7PhbN2AuxHf0bD7egr
44I/O0WLag7R1PeAN6Su/XYxlHZ3xhqIpMydr/YQJrM+x5RsSzOLh9SlYHdND4Qk
HlQpc5sNYyTvvKdTSBWO348NZKxY1Yj4rNsZA0tqSDuXXHhMM3X7h8Kj2DUmzWf5
rNyIIs8tMazdTd5EBRT10YoV8eajTIZkfFmTlbJGVku+89tZJbgv1J7sO2fhVgZn
S0YuFaWcIW795iZGsbwK84pCoZujEsSq+K1iMCLaPXpO3smAIeHwHY0mYWXsnA2q
RFQRQFojC5u0F9QPTxdV75aViA+exsRF2J1D7nHZQqb/AUzeBQgARavOF/XbB5Bi
yv58vdY8M9HwezJrZ5SxfJCk+7Wg+9iWxRy1rQK7gTVaKAhBphS/xmvzDkU/39hx
/2lNLJ2aJM+Xo1/qGPOFMp7ps9zqkp+zfFTHok2WvxsUBBc18ljIRGu+LnGm6zLw
kBA/srm1ISaMJllPMoQWIDDruZ5tqMKtHqVOmD0tEnVp3CTy6+lcMmarXbidDoWo
Z7a0kkHHQBbZGzmqZSqOGYKMSUFEgiMqJs4ytVnUXua1VRzSS86YKUpeVvayZ8q5
9VOjG5stWd2UW1kjZw5rq+FJHFDUS8TK7BqtiKJvFBmuQtstEoTLzW0aSzLmA8is
X2XjxX1wxP86Hp1iP1Q32Skhccy8VRZpyZihiwdUtNaON8K+3FV0+/y/BZQUKbCH
a0+Opu+pXyPPvNHXU8Vm+8GwpD/edvG4dBa3Yo4enbLiq/hywbm1FMgmZQLsknFj
1zhMJV+/UIj0O4nslwX//t+Z3E+u2U3xQUDMOznArruLtzVe6SqiEed3S9JAmvPZ
asUEUedDupWr8z5SlXlV6LkfShb1LW3jeX5gL93M8xVh78DbrJfuIu/MfZl89yB1
3z6xPAf/+/GVU4ZTC4XV4S9oiKHaT4ufTltMbFEpl56kDMxrsT1JvJiXfge1EDqm
JzufZScnDZvrcFLVEEE5w/S7t5LXq5NFHWmGUoVk7JReu3CNcdal8Jt7oOwwbrxZ
R1/CG+K8APQX+aukWEBIWloEcwuJV98zOo1qsilTJS+86LqTkvYNfV3D5uX+cnsS
h8pFpEbNvqUzLrTGdg0UhleoIiF1xkkOD5Oru6LesR2CLS6wZSKkNwP02yVxPxQs
a5E6Ccnf2sLJdHJoo1LtLua9L05bvwwU0NWVgulm/NnKAxRKJbRTy5e5ZCuFgw0H
45/6ELspPofSsVOhz2OekKFWzcKOgZ8GHFY0JiBgeqMmHmFjrGPHH2BZU30j1i07
A02JD7M4CQsPvQI863amUcmQf7Vv/hnVi1GI7t0R+P6g4JK+L1SHqPI4eWepHbzW
BMEavLp9MNRUAtNO8Z59PRIaWNTlsAwU3NPg9HNNIr5Csu3wG/17qAVtuuytD8L/
yegmcfAHfrLS3Ku3bgJQsuGOeMCh0v6PF6lqQ9xokMUby2X08SO+M0Tvr5QMWBad
2VCMOTAYo1PXECPyI+9Ocd2XJL03Cuujz3Ejzn3U50Q1ljC/AZLS03TAt9oah722
CRxgZLppnEFrKn1o6H24h32yd4Kpnw/bEopsc0aZ2mBWnPhH9+t2NkFf3mdNKyIa
S3t043ZB4o/L7154fUmoCVkgEKNdmcSV//2U+3JNrbfkwXBIoXdUnw5OwL7Dhzgi
DcO1p1Zq5Zd/TbMHHUNqxliZQkR8h49Fmw0q/3cp6QdO4LJIMjha+bEDsXzVAri1
Art4CkVQfnztEIm+V4NwcmRWgtUGNMZJ3npQ/ESLXxoU0lRHxbrS2UuuuhtSjV1q
c5uGwZQ2r8cOVXYKBOX/g9ddUv1FcY/OrZdj/MkaCw3jV5llFSIaSfzSJJERlCUi
jMdpyrbEWgHWORdpbt1hc3mjV/zyQUdKFQgqmV/0Pzm0jsAHb7dav1xNEyPON4yh
5yLUHmPKvvguHB46YbQhWJyL1Hgfi0mGAJwoCZtW+pjJo4GkyHv+HDNB0oDDeZ/A
uMgRjLtSOPwrNKJRL+jNSp1kxq56aMSHk6RYmF6Tcd26NpjQx/6ANHxQQJCHnIX5
gZ/BLg94ZcsRxkYoUlNes99yQ+6KcMvZfw42Uz6tJXs9wthREpfn8RadPUCjoHOu
MCZD52bylqavCGyaTHtWNrZTH0plBHD9uUVex63C8+eCwPbR7cdvrtmOWg2QOByp
+tGhWTTbsRAEP8sUfrOFVliAFBp45A0yXx58Sszg0N3GUhWCCLOmqdHV+kt+NAZM
8fBFBMu9NG2VsEFaTldXjkZasi77AnV9J4HpZ6NCxbjQgW+qj3StUHeYqP/0nxyd
US2LTzQbEKr8iRD54tIE9omAl8UzT+tWgKc4PN8oQcVWW5KSatW8RhfCZ1eEI6Rz
iaMxzL+3QllOaKVQ2/cnakkmhW1hshabj7GS3T0JGkwUh+hZEvdF0G3VSVxgEdeD
DgvlzW3ZWFGTcoe58O+63ApwxkOQNGBP1Hm2ge1lJ9usTvJ1FFM4LVS91vGN9YrS
AtkJtSf3SJYa9MLomuOG0J4fcv56SeopupCgdPLT+R0CFuz2Dv7y01BHz1FinA/o
T1FKH5lC6pMXu0IXzEj/aERX51ufT+lD1Vpg5H3PsjNBSRcGl7prtt31i909nu26
mjInCxGdtbClcU/Ek6wdJ1Qcx3Hzz6rxrSfSgqv7qwU05bZojbLOyBAPv+xXJzid
CdL8Zti/VdCB6W8l+YXyfaQHpwkkLERpOQ6Sq0ILTdN3VDyhr/gEt2rsCo/bLkMl
fZinhDxmO1U/dSs9I3Gr4VDOf/DpfUKkffTA+sGzxopGmufJ0AeLY6a+VuNeFP5M
9NNDkXjoWLm2Xo+zm5Zfo0W8tNoP0jB4Mk2i96p3MO+kmV06YQjZ2reflHRHtlBq
Unic/8s66GIZN0G1cf38O4HBsLr4dIqeek1qN5TQheCbBDVL50TKtE24hBV7N11W
B1ffCdKMZ/cVfXoTsouHSDX+/cPrw8XJjIhngzI1ihBnRS3D0u2T0PiDrTMgaHWe
XK7fzZqp86Z+AZsG+yywVLniHVJnwx76C4kzXcF+pXn9nbXF+Fdier+mFD9ujz1a
fqB/0HS20QWzypZlhibdsTPsPBFVBcIMpTl0hkSu61IdvyuK4jdB1tnj7fZ6Cda1
kcy4uTO1+C6TlGHSo4vn36gA8zA3lgl4WSBBdc9xv/8NQYT1d5qcgvYnA+XOSJDX
MPtO2grPOg8nFXFDL4ZbIcSQM5f+GhupQyf7v4fuDEtoYU9hx2FlDwvQdgJH4yIG
nhafaaGGFZxJEGx0ui2Vo+AOOOlDyKI1m8I0GuVaAXp6EPyheJ0GV/WQBnAtWe8X
3193gwsM7g42Mhd7DcCSRAYPkgHSP0pr+VMckK/E9Xr7ErhQqQHuvq4SoQOrRFZK
5xk6cKTfYrGEjBrKqIRNhmdHvp+d4OltUD3LzQQrsREZjXqmVTJbx6EQe+Dp7xzj
hEXRzzr3MmDkW+zoVNd/ES/pAagcPr8edMf5FFEDxX1TA+L4TZEAjPxjrN1wo1eH
wGfLary09U42qEwvdzaKUwwTkeIpu/hVddb3W/UIwmRHJw5+4Xdc52VM6vSRhqEm
QAWNOlb/XHek9+jeYaJqMLWjO56ob9aEu/YqiK4oDJXeJuriWgVGwnoJ4MHJExcU
v0OyX70bsYcPN3G5AoDRypGV661OtN7t39M3DZaqb+VsWnPGO+jmltwx0smFzfee
0zhiu3hlbtJbnOrkGuGpGNWA7KiE3l/KBLVGgP9r4egJhoEHsgmxPl3UD9VxHhOP
NKf6dqHTRm7qY7ayjVf6Zd5yUKvdBHEvBm00XL0YPi3+Z0ZU9vFcEet+m2BV3lu2
iJ1RsJdVcoCnf59UP4AFO89FyUdKWqqRH+xWehLS7h3gKcsRWBdhV9efpTuRIttU
tDGorLn14GutsKUceWTtH3BTJIvxXXhiqAYuGOP+RuLwpULox11L3Ojj9dkD257a
j2aC2Z7zfgM5Q0ake6kxaFt0lSCfBQUNeEGTMiE8ln+kbFEAYSxkySm87INYCKB5
oxi0/1ThoWchN2ns9cYbQt9aqVX3kItLcAl1ryF/OWPkrvW6RczjWAP3OB/9XqAn
rBYodhxNOzIrv1Z+oW6qEXDYAyB0nBVA0lAXo3CGGJqRF35nGjXLGXxyfSfdlo8S
4ajj1z4VLLyGV4jZsUXi0VzHZDAFJXMHqBIvv8D1ZMh07mvZszKDcWI4ZD1RfzLn
IILvaHKvlE4oukvWAOmd7wngMhpKO1BVqLUobMIAWxxdniF7Eya1E98rZJiaspxd
YuqznTlwy4E2uN1agPp9OMprMLiXt6UuYQjSCQH6qqVrDAjlI/1DpTYur3xpxruX
UZzCIIlRPisTeSL8yBwhpQXg++U8mrmAhxhALmxgWW+G5R3uHUb0665Z5lJV5Mcp
oYLBNbhc0J8JJifKtEMM3SVc4RdM3r7WaFKFoD1drp9Qjm/8pCRZaRl9Skd74Y0E
1h5iklTM5jZmmDDudSwi5VpiVxmsCxHfpekSPRDJ88C4F8tiDvTLmXSkTTSmvoF/
qQqeq9m+vvl3cSTeT010yehwcQMu3P5ZmQx8120QUD21m/PerFRZMb/tTNdrn0Gh
tivTKPgLknE/e1XVyX5onb+rykq5IKt/M80ZkHbPAU07Hn3nylph5Rz9fxH+hwU/
7lcT1e6WqcRPkeG5Wnwqg1+Oo+GlABERV+SgLqJnKck8mqAQlfAAHCEE3ApTJUMr
i2ngtj153VfCXO3AV7r4Jws1ukhxm8FBruxXH/QrTLLXm/nPLHUHVrWtzoolCyLn
Q6htRDmFLN+IBBBMHVs7Jg7rp2Shi3KV2x9uDawMnzTkWChZa/Y6EvEZ1g+iFmes
9v1wtOJH0QwanAN5H2ksjsNvUzsMBUM1B0RMZ/rpVkYfcCkiRCDYJcLCgM/Vy1Np
VkbZi15tjgsVmkUoa12oJLuGKjo0uYwj+LjWDpz+MDEgC/x3lXk3Y3O2Jnn25vKu
TMyekDQhlviGwRh7Ikq3Ok2iCOat1twFKmwzuRiU2nE2FZ2BsHmTe4fPGbssSAaE
IZ2/nCzCy/FirqOV5M6drRf3vR0IAOA2yA+aYiWf7MNq9PADH10pcJ1KlgyKxF4g
AGME8fR/Iy8mj7/rHUE8WKcx7suxJXOsAo3sRW3CI1SAsUAipjO5QynvzztPVew9
QD3Y/pv12F7jy2r4gjg+W+zKhQPMe5ZEFUB7zpknTNZ3mtw/XQNeHxJdcFyRovZT
BzdvkfpiziYLUc5EZ7wFAyP7NuF3y4PDoe4nXKbgW9+pFJx4HjiUDCIBGWRVzOfh
Tq9Su298cWMMliSIJhH0EiwvpB5jKnUhIan+eWcTXLJbikThORjWi7zpowDdYPSC
jT7iefAz/QEGrMCaS9/ztTXtIs9u5f1R/FKzLebfgVZNkpiIIbxe/H9gS1zaMr7v
uW4WD9NsNix8HdoX5A6Pm+9ml53Ad1Mfp0hzwTn+t9J8QYxjIq73haIYVzPCte6p
B4HWf2PzDw5zBQ87GQmlqDWOOUydfYC/b42YcL/iEd79RZtEiGcIT2VkixltxD3G
5qWW1JaXSUvzRk4L6eMz8COSHmVrjaU/aShkNvum1D5r2F3wn+c8yuwqeHCN8I//
ajdeVtbQkO/UPfeZgZmzhTMtk99Zxd+KTJv8cz67qEgHYqebfcif/ycBfZmcYw8E
zrJJd8BoqBn7kRVmMnYwS67nTvzld4Bq/rKUWnXKsnYu3f+3Caw9EsDSSiohxY1t
YU4AKJ2E/NASQzgDF9XXd0zit4/pEA6y2QwRajLxGWJ9IpEYRkZAuwYlabjy4I1q
xIkGoTJ4MF8hT6DHkh06icG/rHj6aB2yqTDz+GUWd1jkTPKURSNRkB7jnkIxME1I
EThEXIOGO64T44T+3Eegj44xXnFVZXflKhjXLQ6Ea4SInOFd7s5S6nyZtnfLSKUA
x/ML+UgLtB75Yok3sbwmsziX4e6stDez+t+TryiLg8/fXLWLn1iwzCrwXVKU6ihz
zrbgvJMRd74w0CgNPx0lJx5nJcXko6bc2x/dyL+BH6CHL+eyKkzY4UNQ7lw+IWR8
DBfn1Ry5JMfC2WBJQiRAj67gIk8IHgBN90FKRmy0kwTj1gCUHN7PqkUl8IdU3bMI
OzQ1PIWotub4BF+jACNaSEYw2jBCdUE5Bdk3E/jB4cI2I0MPtG7M5eDkpSPT9wC9
MWQNpy29AAp87DL0x57GeO9AVea0lzYrvbA+vTk97zhUC30uP1dXr8HP/Kx7G+Xq
zbjL4B01yw8ifbDDzcrHadj7N4CPOKXbmKnpLSrSY+kSF72RXMSdbYc0nIV19+kI
9h6gLdfTycx4TWJ1PGTLwZucSfZcDoIhYcq4ZO+lW36/YzAjnkoDCh2YdKhgF7mC
h9aphg7iAwxAzeYG2EFnyu6sbiraqrKcn/TiLYmVNGnPGVKXddhYS+qTZJa1VtTg
hdzsQ1o2W0u6kPfjDU9MqrD60uvUOAj5bKHIjpAtKtL+rDAh1eynqANnx1oYfeoB
uqhjw/QpW7dGB7zX35UTumIbs2ctf7bUAhhM9D60NaPihED9IfPTYekSwj2w0+Qo
dM++bjp7iIwYC1sMTYuh69w7Qi7+QP5O+f+ovNwEiVrBRpAG3GrFyYDmoUIp4l+k
JwF699AUw3uZpv4nts5X+2ocy41dqWBvE9QVQHCC0JSkIr8tk3AO0UuoCPnVYCd5
tw46HbgybK3jl8LRudeOaOJmE21pRFOhAJOZLd71agNK43wDEuhzs2QjFNfWGYmN
GQOqTih4gcR1kS9ldWSKVumnaqC32ZA+LSZZFHvcoJoHNI/wcMVDNHazzknuwTgF
bEDZ/ZU+6a8mGSgzhDW5818FA2outOnFhF6dB4gUFdiXXRwlgrnzUyIv7M1rWXmk
lw0UbXFWBTikaky++pvkVRljOJ5Vf57XRCfQPVSn/CiF8yswhMcLRy6fOUtWXKT8
+enz/eg5mxOj1doebgqPPUKhFjcL7UUKGaiMoSWJeP8qwJRmSlKLER4HouwIDZ7T
7nmovNdTOftM5LMrvekv3MuDe6JPff2Lqr+fwmJbODXyGFuBgXD16Tw0rbnsY4ow
lyDkXGb3mdfU4tRgMHj7tymMnILriI6G8/nvxvqSxiPzodOdU7CNkfyvVjOl8H+m
DdKnykbO1rBOnC1qebdOihZQZ0aNdxHiX876Rt3UrH/fCJzh5zxj9GaK+dxEMx1Y
475uH+3UlgKfhEQAxJgDNgUJM6QS4t6SYrcyM0CGjl+ccJ5S5YsJHZQK76Z76X6b
sWbUykphnjhj5RIxeq/alc2LJyD1Kj6gPzFUXt8UvPv2WBfRyiIhXc67TSFo1wsy
Mp/nlJ1afhDSP3cRc0xgErtboog1VxgMGRst4XCivLyd/BznEXi9Wb8XeA1XbIbt
w0X+HDh3Hkr9nP9Zu4G5fBJ/plPz24yHAJxKmp8NF0Ac7GPhf8b50j5O/qHPyBeT
JledrgVVx1Xhhym2RNwkd2vVhAQekYOsz2fMDIC6XsbRkKooG5NcTPoevOJqfAUH
OkjVJ5BBd6BnMbGS4p/DYf5gjhwIQzVW3/Rq43i8BoPKlPsfbZ0EeKOtpgaNtOAt
RsekOK3/Ouyq5WCbUhqNht+a3qxUCM1NqNHrdoJB3D1Tl5TLrYFSEjWFzssqvRXh
ONXqFV1Vq5Hzdr8LSqhl6LPjHF7V4Uyp0XhURTKojh0nRKoLyCheGo4KJco9ZCvC
HFKYLLr83iPIp6za+TAYzK7m7kvHuvL0Jw5kHa+w5D30nxlRKUUYNEqN02sTfNYV
8eO5o2zvw5KsqpR5ho28Gc1b3OOnh54cUcJq/CAgWnXk9tkhZYJcDqotXgkOAuzh
t2ax3qPKccOIAsFQ+0EcuTt+4kGHXaFQNf7HQ/3X52ny6FwtmtHyToQrpsVaWE9Q
02Ky6zs+y/7fORxjkAnyiVxabQr7Q9+SyD2yJcv0BJxGN84oXAiU3LYqHxmKOEqS
jsuqLlEh5SElk01iJVdt9K88L0wjBACSQGrS45G8r18iULcFEihMXMN5Fopjq/7K
9SUkLkdfFKYSwBPlFxea8xn28FSIHQiYfnkbfPURnqAbsY1r4yX4ZX7BDRBRxt6t
U9XrvvB0Kr8Im3HN9dBVYzmosUq41NSt5KuAvVXaCQKeEZS5cyO+qeitg3Jj9A6p
Mvn+SjSHa2UBmbHuWOyVtfNnqTsq7Ua4zc/8i/YhAzxMk3kkeu1Oo9LMFhjPiKfE
IfZVsssD9kt+0708OZAbGGKyfN0/USDT40TSHk7Yqp3xbzbyxaRNNYyJCjPi9op7
HN3bboC+ah1yElRkm5np+b/m4mqpK9RoYsSPRVFfo89ESOc9ZbOFvpsKNRqqlrEO
c9GF2w7qNAmG6EUmaiEZdIV9uRaI0RnCQNCaMC6YyOHkD6xeWbDFRKca8BNxNa4L
36YjGBfO0cevX5zFmnEw1NmiXLHnldy4KUqH3kaybos17zWdNkkmubEdWhI7Cbj3
KXSt0PB2LtwGPIm9zHCCbCHroK7AnaXec0UDsl4REnGtWq1pJkiGDacnDQFpGeIo
IhPCzmRwYeheF6eoPszLvwjxzwLBqjzzFRovPy1H3sDV3zfdUEeay3gWWqvNC2ex
OaYbqVCFdUvZE9VodnlDIhA6Nep4JdqIclscgZm4Oce8L4xy7kie1Z3t82vN98qL
KRTuE109ycMSAMKn1XG9PXYefYyASqzmTQ+QK+2SkhJJR+aS922QlmhH149nVap5
KuMu2VLn4fUoO3WLpKfUo0H+WwVs97vMX88b5KKwwg7xeql4Jx/mQ3FYR7fxhF83
/qlO3T3t4pr3B9Ud+pR68mxtZbS3qzpgBPVvUadmEWAPLZEpP+Dp25HbuZv/qac9
ih/9U6PvllYw6lEG1xoJ8RK+bdsEmgs1fJl7IOgtcfwpAacI2tI8uUFNlfJb87li
wRbWhSyl7st67G8izLcAkJo0fzXNRd+TAJro0pX7d/bzmbJwNMDkQ4407NUZrtBq
Mnu6BeEjD2YDAKnU04gzzkcaEzITfGWeK0SlUDyJWB8qVUsRyHI24F8bq/xBG3os
MH6ezVedz3yVVnuEs+t2PbvfXNv2m1cNwc2+7W+gpyJT9GPbZItr7Op8J5OLBdRe
a1nbjLYgRAWHFt1RAl87CTG4piAvHzyULS7UIfb+2QLqaQl347kR4jJQiHmAvBwo
i+Fvitxj21Swn0WfDXucrE06SaBphTKSa2t6iWdZ8en8nWckzldteRW1IYNcptLE
GfBujxVTgD96AM0rxl1OgwAWIEZ9CEnHYAtg2K8UpTRlMs1zl6TCb7hAnWFdrhLj
4nJ0ucyWIYMixuafymXQ2yrSJU4UyFZkgTv5M4fLNU8zvnYGJk84PBNLXrhyDH3j
Bb4AtkoA038QG1SPDaJ8uBSBa0wopXHYgd8CX+egI3vRCzxxgorzWNdStGUpBWBE
+92xrs3xqNu/nb2sGUiZpfzrAK0n9RwAUPiDouEDAstrng3XtrgNebOI4H6bMVBT
vQwMDgJZOnZsglc4KXowKvZ4uV5mTB/Tqgwx6sEAsm4vFi/5OcICmcrEKGJOs6aE
zHlAUHqH8A4d/DRsi6/0F54S08qeJ294QfusibfhUz1amB1q9KHnRRq3KRtvslMC
WfyxdqH/OLWeIhTS2b1SsKzzJ/HiZwgIFwusQO9J5wMprggCNNygWkP34XB0sLNG
aV4vnQ/NTuOtOfCpM1k+rPbM7zd1EO2hTz5v5XaE3hXrend1zDLv0OCGPQSrTZkB
t+Oykt7AfIz4fS4lJQRZjUqY1hD/ETiO4Ox6oR941w5wCxsiwaJdmDo5/yhMNSw/
m5u99gVnujYz3z1pT1FtJUDVjU/OLMEBFP1ziaECQU9Wjugf/uaZNSN846ZnO19D
IXHoDCxk+k/kOVRUPOQ6EIdVPw34gJDVQG3t1fgQGVu/IjMpekLSp3sFxkOVfiPJ
lVssDFzMpTepZlQbjyESU43yLZBA0gz+3/9hJY4u/mCpWrHj1z/Eurk5v7hD7nZd
cwICV4LBz1yrKd6QUJSLfdv18gExQzD6cb/iaJC2dTBtlkl5a1LxsNnU4bgHfRMd
PImzaxBdAwSDGNNqirzdh6W+xv7DIC6Mz1p/XMaJwJvLN92x4k0HV/m7F428AKjJ
g7jsHx8RsjURi75kT+9U0nkOV99p6hMhxdy+qxrbkDrYmW4rNGuUquWWNpQzFTAV
zxtlwZmX7LpJWb087eWRnx7G+m6GAC1/dOGRmvL1GAGjLXcMMkit15IePwfGWUGH
qQnEZeB+J7YtcQ3XpHlVkxebEHjx8Px87jxBkdxKKCepYH+xbOuz/Op4c4e2G7oQ
cU8WNmNQfhqpgvFRvZ8OlRU7r+os5FSfN92t3Uz49eA5MyrH6SwIo7n/GKrelOt4
sYj4M7S8WkDKp1RmEAjd+r8hg9S4Fr/kDH7mgruK1mmaRO5nAQ3UVM8FMhq2woe6
+PmXR8BhI0ZibWYfTNzP+WgPD0NLIIceZJ8FqLEGcBW1laUCR5rScf3J6tswdfMg
iVL6xnmHN0OQRD/MEaJGwfo3ELt2hl5ImIAA8hTT6My0rG2KFt6NGmQOHmG96kLc
WCiW8sWi+gR3AqA1qY8TfJ6lm48tbh45iQV0GQ3wG/6qItJAtGvBsgaqut2MM+eL
f1zT5YxLt5QgHIrVFzoMI8rBSAKCBENgTm3/4T17q+lnuCM1WWeDK3NflT6SpCOn
/yqMr9PnyY4pnNMtJxUeba2uZGRnyOC0vorF+qHYjELsXbiucDIMDpLFPiqJUVZV
4uIQtq7As+4Sop5VPSnRmF/LToSmMdm6POJWRhM5fjgo024tJdql5btW85ad530b
KpBhLFNCdopbbFpjBXaF6K1WdfV2OWHrw7M6txOkwwvGIxO7gj5sAfZFwNJ9mdLv
61qw3TrTI1aQFRq7sZJ5qUEfLopGdcccYRXAUDaBGfe/dzh25KDsAld8P8FnPrtN
6+ppomEMWGIUGk+btIKATvrjc1Fa0mYJUdvoVYBkADNwokv7K66QOqmKZtv0s4MR
IzAu+Jgbxnk7CXkS/4Ztwx1lMoGM+1gJg6glEMmM1rEzN193MFcRarvy8uaUFuRC
WEU7RLHynxXUkN27e64d03kLZFY0X4XlPJTz2h1g3ahKeCfwa6IWqqWGBa7+NTM6
baqg8hIgnLn0xQB3AOAynx66xkTSUcbV736tQgyPg/PUkeqER/KtyviQ/Fm0m2+L
CQ13SVZ/ilXdA4WmrYsyKJzOyG29x7Wbz3BJP2wWhmvgphecw8RAkYvwJ7CnpdF/
/3Km/WPeq7vJwyuftyb61NjBFEPYA0Q9pPB5aWG772eHUhqypBoX1ilzYRbp9JYg
Eot8brjorLuT/2iSSFyqmxpU5d5BtqZs/AwsAgjbDeg7DAxa1O2sPM/qhLsshoaf
O0TY8L+5tlUbMlxvIT5H4Nchp7FvathI1A/Tf474Af9TaKjoU3LZagLI5mAdIaqq
wog9oexMPomvLNnCTCaNJ9VPfreR48FV9AO5VZW/ZQHBMnLJNyn4IXiVAdvWifku
028/B+8oYZKLjD9owUnK1USCLOXeNwrDOzHso7uSqEif/Rznbt5cHr+bdybhagtM
1FYkzfHEindlGK+E+CedAcaEzKntlbli31MBJJVhSV5CqCPFxLZKqPfuxaCaxVxr
kPtGbg9QmHIDPsI+/NmntlvuWcLARFl1fCIJbDI/8yXnLBzPBZkIp7Z1Bxb97+BK
J9bagA5x7VJs+K9QwN+rmpYa5xIoYK/13z6D27+YGVvo4Yf2VUSEgVDDcEOFCL11
tzMBGmfnxxNc3vedzTlUZExvZDSjTYbnXJa1cu6vrZZ99Y3I/GM96+23RFx0hv3z
15HtmxCQipzg5f/aepte3edKTpfsq2Z5SvQIbY7vyeA4Sqm774n2FWRyRYoI/TnN
euQmofwRnFRfwKhFYHZbo4JQfuFlJcd2bAj6Afk+SWNIqpkJ6K0z6/WN5BO3oJKJ
D6n4AoTd9GTgg12dPA52++6Qki2cc+gTspxjSISkK26aHgmB5dulCYledpmETve4
vtWYfNiBiTtBWuIDOYHPazM5u6kGbMCVxxZX5JOkFqbIznLqn0JfjrMYayUX4reu
SvlkujUupne8LAudT5FiZD3brOeRjhGrGpwkn1AJ+l4KpynYu2AHA/DQVHU/cvXG
0M/lz7N++mPzSVdmtcq7ZI2U/Lai6VYhciyptoxbpK2AiZekZYLor0rnZPVtImWi
DHZV5okrv2eHxZYVIF6ZxWPtti5xQNYaou4AlF6aH6yGvp9Ihc285xKK1v6mDU7P
UgZMs6D/YBb7qyBn+8pG0kLgZxPprCDgLUyCAxl5zc0AL8ccPpRllz69s2GROXu3
3j/tHwqcbH1P7J1EzfNcSGmzQihk4HLWJ2s0GcntLd0wuRNMttCYxSCkcuoZUxhn
/XYkw6s1uPHf9WAhPJZzw+xjwKdqHqUJazkj7aljzUva97/z6UGNnMLihWLcenxY
EOvo9oKtiKB7C22eFLbZHfRAFEIEPH7ucAGmM/gGpazFVGPgw+mcOFnMUFxzA+P0
9uKsa46wz3bhzsXjeNOMgIBORU2/Vv9O55vmt3F7NXl2esv5b6r1TZ5n/xyUSVSX
6OvVOYjkLOA0ZvUHzeQ4bPHih4lNLkL46SAGyvHNyGilKJcoEZT+dKEhmfG/RI79
2gbNKTkJNtvMbc7ga3Fcu0BH+IYBFGi6KFlAlrPAf8GbgyE/SzrjddaJcRa/KdpU
22uhPTm2xjxPtYc12RYzT2JeIdRQqrcGm2EVeZ8fzywpwy/b88xTTzlO9j3xAF0w
ylER7aBS2wB9Bl6SUJ8mdnvgUo+PY//noIk4UzFX73m7RCig4zI9JPNoA1o6sVrg
6lA/vhZNUkPjUifM6/lTeiJ3FqkDtW29RN4DcEcc7VYI5AM7nlA8ravmZd8LLswW
A/hIayUVmeJc3mMaqSH3mrck1NfgKQha3BemHQNY1jhbBQCTRKqfXFOM+LrbtkWp
NIJz73I72PrrmMFvTO2fXDfUdj28473R8t0XTtRHXFYtO8zrosEnkjgR5aayT3XL
mtjvxHBp2BqYDcynJGZlJ9GPwSk8yA7JEql8n6SFRFyYqk+vZi9gW/P8vcyqQ0YT
22C9nscA6fU9qzKzyvIPJ8f3y2hpipUdDVsOWuYWGxAuwsZzZ6tLRzn088u/KdB9
WawstwfwqP6XXoIH1iXL47189Mb8mz748640/Yj1p5eFm/M16WkwcGRdQTzxpeeS
TanpZNXA+1+58KDQXeG8c1m2PyG1dSN3hyTOS5LFXel0aHzSWzPgmvJ8pp8iVevV
XzpnP93tK6lzCfU1tm5tC+lzxlJRrurdGSo1RszdpGCBQ8bGsVWqrC2kyiRffYnD
ln7YwYxAqi2gZkrsw2S1nEMPhUmKMlutqmgUqxXcGSsnQzg5S2TtzRhJhGrWeZWx
7lvR+TYjy5wm67+mD+ODPzLYuxzI4IwM1v5BdT27i2Hmj+SYaQLh6OwLUg7MMz4I
g+nUtrc8vwdjalNA8OdF8yHNLf81gP14tPFTDXY9Akmbe7GZ5bAIpBPFIPXgjcKK
v5QffFTOJOTVWbv9UW3sqnz/Akvr2EfPXAWlaFSQ5lOV4R7ON5IFoMfUkSyjbGlc
HJMvEUk3DEziesKrbjxSVQzL/+b08pDlAsTqyCmAEVkAHf0t1jwN0i2siMUFSPO2
Wq8EuSv5F1WDEKTfJrCPbmREs0sKUKCSseJEDQBhXTET1Mg2r6YZyEw4IeFVpOE+
rWUR5bRJD0KOrf1ijA6fsvdx7x8Qnh/YcQ5lps32cQNvYO66vlnHThW9O4JxurJ1
JzQWySsZG2T+bgtduUFVwarIb0LXwjtOheG5c0pBz+3w87rWSP1jeZdRwhGxB0xc
j2vkrRyCEIINe50S+fauGgZ6R4Su4JubfEwPn/p/bgxDXrD9sUdDgn5fQNcWCpZ1
5tBJu4/hcHoGqN1EyYWsG+KV3wCPI4y9MxBe5P2ba932EUW2YrFGA+YXp+0lr4ii
5SlyeKEkKYmanC9qHdH+YUAxUmDQnx8F8s8Qun3+PaJwBokF0aqLtyhgnE8DhUDz
OdzWZ4ZSD0SVxUFs4bbbqEtQo52+tKNGqR6ugT7qvVp/mnrINGAwmk9tnW85tnsB
+D4h9Dw3NO2Q84qyZLR+1w8RdTSBtdncCGh6xslXNDIjZw71Ccr35Mbn2oE+l90Q
yNij90eOYK9D4vHsFM3oR5GA3scRzCL9THN1T/ix5ekOdPPcJ/WURjipDR8YdFD8
6yzdLj6adWPRqIZGPJhAOo3aRUp4+9H+0PzKNPb8ea0RkvI9Gl0MKG3XkrwfzE62
RUlQCE1IR9HV6G/H6zsCequbzTN6SqjDGFK/YtU1ZAa0vPQdgUbRPpRDzz++z31i
iXC3oP0VPZfzSR/jYGvjTAzxV3FAPigrqFENR1b5bM7AU9IZusVpijHAY8F1BA0s
yeJUfGBCcU/qfTey4H1QPi1nRs87jPfT88lG/fX1K7YSuCDwpi+0TBJTK8iU5RCw
b94j4V+75k64w2g80K6qcTY5Kid4cGBEGB4/HDXUDI5SvLKfvr/AOVXodNiCf1ua
N2/aUKqrnoQ1NVHPDxajOiS3RjSlbSirgNPQMdUnynyjkpXCtRrr5I/9Zr29ORNc
35haBCq6l7/RzM+ClIQa9ELUqljA8OrwE2TYUZQTCNRqVOXJvGO0QoRAy8+OYKND
iiPOytcKXvXgijsm1CWbB+FB6SpoCdnM97XdcUaeWn/pNdVIKfbN9ZpyiLIW/rjE
kVAh4Eg+LyUg9IDspdNVoF2m0SqG4XSFHRJb2T60U1xYeMsR84gU1UyAhUMFrUkj
HNzda3m9sMm6uNSTIwEg9z9WaTH2DXBSHiSES6EOTH1zMwoc+cLHjibq3BRP9CD9
5ba1Buxso3/I0MCCOgPqJaYz8XeEWQsNJIY9V77ENCsXlYh3EnheQlfkiRQkrdj5
/yREAv3FJVH0v6xqfRYsrcl/CtB6mLHgunTomOIYWPo8X2BIe+nCyrafZK6o59ha
ndoUn44+qH1zi6/nHn/jwX0w0poD5eQsMDnFMdNkm0Z8FQuGqg1EGz1BPArwUc2T
ieJiWDP0XLJnN05E4YgYcNGEXk1yLXLOsfgnp8A+1k64E8LNu/ENE3WrSy3UCIGH
EBoHOabqinqcB7EAPIxNThi1MsM2I7NZQF9FU2nsb4p7YeTAQ32bDL4cLrqQCOBu
2QD5aE46CUEj4cSnj1ktNQn8xxA4aaUvsQ57yqjjfSz3ibyWWYMQvluhJ8yp30Ob
6cg/UMvqGFolLaS2l5E58ZXactGJldWvsWHMJeE6/OWUO6TFjc8bkUJePp5LF35/
Jr/BaeTBSGXrZdcAe/XZGq3L1rURR0OYoOuwaOd5Avr7An8KCmWMglnJlaI2ZR0y
eq+6zhsEwMl4/rpCBtt5+dOnwMwKXvM2xG7FurYl0a49wLighyLRLOWEJFjnYhrZ
KGR7PIxRyVErnHefDjeqNWoLogKKWlJuYRoKczrF83MVTh8GJuxqUe6ctxdgisxa
u2xCGEzqmiWUfjIKa+osvHpBKwyi98Zu+M/c4pN3YNsabe9MgzgFvweOW98a8zNQ
izyV6jS1azpxzyxh0lRUVzuigWmVTEwI3G/H2yO3SX4dMAHebJ8GdCQ36ndn7NTW
BdvtmhjVhiSK4Pf148eEp4qSkTIPmRfKa9pgjJGeG3iYdiTYOTFONd5586ZJvng6
5hq1kZcezj6BQ4SzTs9TciZO2Navx4FIEyC5CTsrniTNflNzFRXHUd6Rhl3WnGOk
acUV8VAe1CJbjOpH2sJP8lVsyWqO72RFHqLQNUGcrs7ud6kFdQioz7adDKv/178d
Eeqtgue6gu/k5MAu7scTodpIKlK5yLy3c/7AIPLvxqxtu95aukPJKsowdEkvWBkT
mTDrdN0vEsNfHE3vtqdddIyuFQQ4DIMkfrTSy3ju2jMrV2W4JokfiuAedrUUUweO
/tWMoPQvVh4sxMCVlNNGWOHdYgYo1HCGuChZqUQBCDrMsbEZybgpnRnCxJLm4/w7
rcgYRYDslu1kBjTWEy17lc3J35ZwbpF+hjYCji3ANxOK0b2/Bqo5wjUh3lFB1EN/
BHBwYAmdRQHPMxM+yMdY9TGG3lZh75SnF7DE78gCpuHXDCS9yZhssg2L/5TczhVf
kba/+maNrl5SZwFvPgAQYCKNQqsXrIS6TUx36y18MkSCMqZz6I8sq8KdzD/Cfepa
pT5ZvjemUN7sshB7fr/uNp2MLPvDDARvhwHABaK09HhAeic57/XSn9PisdyG4JA7
d8Ayb1hUxg/3CmfEz8A2x5CLh1R1OfctetTLsM3YBTKsI2YgkLtlyqc5pdCctljC
FYaC8bWiCw5HoHohRLraPt0bTdPDjCnVHA5AZD5Av7W3ETQeZ5Kncwsxw8wZsdsu
5P36njKxmuqoaey+yfA8BfwTlBNL6KCr/Deavtla/kCOv+SLQegJ4erFJebSY0XS
8A324liOM20dpBamtWDGNQ/mB9aao+N+guO2dALNL7/y+6obj8wLXK05xNiXjTCK
A/Sr9vEE0u4lxv5R1MNgSPGniwo7wx98A8n3JWOAFoRyLoFxQUEJPF1iMzvCHQ3T
YZu7FaLQGu/KvHUVg0SMV6Y8kH5qOFneShEuxGPrc1WzvVqdroEscIkyqQjGlEAa
ibvzJx2YrrluiGlkJswBH9NF0D64sw7jCB+9pEMPjx/0WJNapKTgQFqvN76AznKM
YziKaLJJmKZ8WdARhuR/vpVjINIxeaRNvLGTxl4btSzRCTpoTaMUjPSPlLq/VgVg
CMuekmaTdCqXw6JH37F4vbggyuF4eOZKC/HZz+7lcg0MIy9Pzy17gVjtToTKTxPD
dnCN2wP1GjlIFmolPGCHsPi8MsRRLtEQQm4lDn68dW09ijTv6zlWxIq83k6+jGsn
qMrVQU9KnQ/SRS7AxOAXoM1KRb274XO2IlYYeD1LXbCiCpq1FT94jzEC9ihXSu2w
WOfqT0TU0aAtFGIl85bGowcf958LessnrhWZJMzh9OFXyM1SJR2MBmgMUwmUnoER
/caDyc5bR2sloUa1xyDAe7GLeENDUv6IU3kf2x0iP+NMCptDz1Igx8aaUYKRcjFq
C873Iaro2/sx+5+ozFdWuF/dEDKxPhB+2jEuQSJMgkLAk9K8XcL+CkGwtNiAV1vM
QORwAOgXoCvBJMtDXgUnI5vvBXvNzeafB7zh58ng44RcbeM3nXXG53dIfWWnKGLS
N9By1IqwDsTU98aC212FyQYApGV3hp+EO3pnGqaULKUi6igpgX223eQhFg4euP6x
XfTwBlcaSeYZ5KVJsgauWZOcV6Q97a6jEjNYMx81Nq57aJbFubmi+pRbwg2Yyq9O
UVZ/nR6ruhD0UXK5Sq4uGdP49Xc12fu7Vb56h4t/+2aZdpuWYLRzRjEtCEtqQjG6
+QrHi+1rsa5f3IFkJIv0GqOk+hfC8T5qYv60FkpSTFkDVHLKTd7rKlwNiPQst3JJ
Kjjif/jSE+015wPn1Hqj+OXRG9AyWw5EFpho1xrxKFueCDF4Cgsi/eglZODtK9BI
FvMvrZMbAo7KD8rnL30CmB6x+FbDS+Cd7I2krPK8TKyv/5iJaxRSJRgFow7UsMAf
rYVCo4u5Rca/F3ar7oo0VWIW1Ro4uhuyxNElFYITxDvtJ6NtqlHASU6wWC8T5doj
VH5aDv/d39ym4O0RQHVRSEdKmvkaE23BRUqdR7wTmKfj5aNbbxVsymbZghSBzM6y
w982G1NoeCkmqr3Y+GotJIrz54uk93FbdrwBro+zFacFNEMJG7CN6X+ieprczOJG
VMMS/A5TzpNiq1YiDLltDvzbHmODIfpld5F1g4jVnYb9W7UdIc6MnfekTkLvw1ha
qYdGNI3pYjwOm0JuOu6+w/ascLqIeNKtYek55SaQ6oDJVemrEUQeAAbYovsAl3a5
nkMoEnUCKaS54Qp5O07If0otlCUSXGbA5OAVhLVV4AhFQ+zYrfQP/ssP7DnNS93D
hmfkGsHE7lKgaG8HAkuONGmtLgj5qI1yHs8mFQ0kgg31Iu87SKO/wrp+7V/MmJfg
ViRjy84pMRDgk9dIq4txhH5bSkROxqPuL/RuwYGOB3HHxyBMZklWD9lFX0GBNkl+
GJnvOE2k2BhZDJQ07VdzVzzVEgnUJcMuoTaI67bAiWJYzCvYuOnHoA5Hsfyial1a
Q5ReWEtRgTOMaDPZJqGvbfSOK5H9Mjf+1miawrOzFV4ajq5GLBiajaNii3vdDFQA
zEitb52N3vD1JnyFohk+mkqsKtqEvr5o6Ai0fBf+Ix8miLgzpLagknF2BLndwItS
emab+0qjR/gRPKppnX1xJF+tyFVMzv+kXPELS+bAQnxHfL6t8LJHabltg2JPVt0m
r8mFdD3yNgFp84z9mb+ICRt8oitgMpb7WPmMHRYERzOAB/n4DP6C7ecEc74i4HYA
yBOpQ8GmVnHryMIK3yyGFtzs5Pjh/oKgBb96qCU5dsUHPXuWf2vIl1A8Te21eTug
rFYpBZxMfSIlQdo0z9cvL5eTo4/V3awHnLjaPu8QDMZ15PlVr+Xa5zf8W2ijzlrt
EQcEeAeA242W+IIZhbQtS0JrO/uxIveNjfgzsWXyx5n8fLSmmyyOfW13yaG5zSDb
16uMxeAinh9fOE03wQ2D+2S2dk8bKMda3aNJMRNaAhsBV1vVXTijd0Q8WdutdnoG
MaZ1nWLc+QEj/NStHY0Qg/9I0BlDRRhRdauhQ/71qFiPqhA0dVRLl899u8TlLqFi
xdjEvPYReqaM2UH1LQrUspC6WR80lOFlxyNPGweTekPKqDJJT57wbKNcyqbGFPFP
HSiN/2KFJnXDv49b6jkglf85QdOHMQn45eSJJcLJeARZAaQpb/MTojZpZxECYMgM
Zo06gTwzKd2++e/2EiP4q49qbH1S9oys7Y0jxWM8m9BxKg0RBIHRDRRcqFVKvl88
GCfk76L37S6A5H411/O+JADQB/egI4iK4XEFD7U9yTK0moFV7JVRFQAB3LTGbAet
OxV8RRP5dXlQRgAeeyI55wd5NTIHo8bVzILC7eFuKnbSXeKTSEqrkiDj1d8g7bHl
IAu7kXylH47TFGEojNJMk+H4xoLPBjx0SpBmVWGNcJ/4chbIwDK7dhSjPVg6ja20
oEv2d732kXgdR6oDsLRMZX9doCIMXNZetQVmn0DtAmbPzF4pF9tkYgzPgzeceg1T
H+l27wBFMAcCw5YQv5ywwfzO8ZXvxPUbX1Md6y2lcL0gisgTnwjnVCmXUqAU3N8S
8BV41zOWMZBDi2GNCRp+CbcwkGbopRJc5OKTGeb84MHuPz4IuqEL0G+E+Xkur9wg
MySWE985AmXUU5np5R2xCRogxdGrWftqh1JTU7YK7uvum/0LiFQGLzgRaZF3l18C
HhGUiVjqMa8SrffZT36FhWezblN8PdzpiGCd9BaEN0ELNKbzqquIsYPe/6883EJa
a4ebFFP8cQshjBtDK79YzZQWXC+wXndRJ6F6b+FxxNMlhHyKjbXj+QDQrrjrFhCh
SaX2k0rI03LTNzchIJ0CF8RciTGpDEZc3lFGDUqF8/vkdo9Ll0U/GkLtnC7IcPwf
i9NU14zf44fnD+UYshlWiGVFQ77hGVL+39Aon0lewzxVvFHFDKGziE1aZJsCe3T6
pG/bhg3mxYfIOPqHtwBsJTwRRQ9u06Htc5VwLliKaMccF3n+DcaDn8bRuIT00A/J
GSiHWGgllslsAkfkhL53LE3t2m3OBEAMk57eij6RJbfw6SFLnfQJ/AghSVOoiTuY
RlRIwppajMVGdUuxGhn+ocUsdJMwd6sqA3SG7FmBVUN+vOmp15yY5NWVpEAx+iZG
V7Cgu/n5woXE2dT+ypJcnvJvS9HzcUqTUANM+yR3iFPsp3UDSs+enRPTV1cL9GUa
F/uNeFrNZ/O4XBs9YCvxLL9yvMqUrB4ExW7C9zbK/DFysX1QEd8tb2oLJGKVsgk7
YHTJviE7RFYO6pwjsQNrmWOXiHrz1KBhmWzFjflABQpDH2IMkqc+FZxd11eSGdNV
KNZmdu6wUA78lfQ/DaQ/1wPBs4tSlapRqZmXDl3c0ZK+O359b6Dj7ymuSBVpsPT7
2r395Fv2d3y64y/yc42Lu78if2BlxiiRK6wo0eqe43Y56L6GCKul/xfJXczx59UC
5rY1NWaE6PSzO2qWghyfboziBzTFk+Wim3ZfLqe1XKh0ohnuldRVHSjC1Ohkay5v
aThUseqnIl5E+UO0xIAkK9ZS/4vcljJjLW7m7nQFC1yZ+YHBdUhmWvBRJX3YhIbk
C+ck0Yl6rojEV3Upy4izVUwmmr5hGx3AZijZ89WYMxZn09uVZ+afWiXQdIDPfKWS
MjbHfI7O+oG20ZwKs4BcOmeF9wtpQ1tUpUOa8OFQ9sp22RA7bO4lJ/Z9A9UzgRwD
X+Pc/bEa0+Ew8dBagz/F6F5eFynxO6zAjD6URdWWv+LZ2qXE23ewOkMzcrxkn/sg
WxmeJC8J7U2kGDPhVuOHBw5oP1kPdKdLHHymv/D9Z6XD8Qv9wJWR6Un9q3LHTyMv
B5CammD1YlSY25SdKhV+VARagw8K7KmiNF9cFS+q6bDNipaDjCPhmAGc7qLfN6K0
+ZN80elrrPgB/99PizzJHkshqPiIPljItD+hLlR3i85whjlwsiDRQsKGPBU1/xfA
oqzb76nE+5n5ev9taGZJXNJOxLzpA4ZbROJI59rf1CWbBcbvMYeN+reCw4JHoyhe
1avOHMiUdhmJOT0k5hBKm4WS+75pIRg6C4c0ccHAq+wuyVawEptrlLPGQC6+qqyq
6RFQpogvkuOEJtWYk+kdt1JH9/mKmID8ZXEVViz6pABwlYVmMQoaI7q3hvXgrmXi
Wtkabc+q3eeX5ttWzzQCvzwwAPDkbzFDi3iujx3OwvMvNEkvr14I17t895D6EF5l
u1bDpsn9UhGKL99rgUbcua/kwtwXZ2yBojfgnYTtJHvLdcegb0ttmsI/BGiRTDwD
r7mYIKji8jbwJ3OWOih6QCfjC+OMnyUbbZNwL2UBx1kTS/A4aAspA1d1tltveg/o
jBffyGD2VuBuNO3NVOQ7dx4GAj6XCX1ObxNaZksR6UgP5fxiTtClvcG177M9ihMe
O+QmmmFchDBXyhd3YE8w67fLZerL66PDTAt/wd6UsLIeTgBobXi9Zj+2iJyh0Q8u
oboPTQyif88EA2BQ71rctZU96yrG4PTxOdMB2h9Xrdb9VD2N72jrDYqB3ZiLNttS
P/Ng5EgpTGFcnpJkR03i3Z/QpOpu3szp7duU20zz3514VuLM34hU+d2nxx7+A6i5
U74KeVPuJ9kEzrUo4UVnuQ8+P2rig95lZrlWWjZYW9Kn+E0hXLWnPVvMvwIDwP4Y
BbP9RKU2Ii7/sBxK7m0e8oaGdABaBaQ5gWSY10v1ye5F3NrctEOaFJx+1AsFYadw
AK9HvrMZYm3mbbkI8IFiXo3nBCeaM9xrTo6SGNuFhMSXmSrE+9e45PfUO+Yt8OzU
ivol39gu/rp0Y96+t2gGtQP3HExINYGquZ+SEifMnHx9JvJ8asedJ1XnCfZ4/7FR
Pf8tbJQ4M26PnnF4mcXBYuCh8IXLAsRgZQbOLB1rP+zyTYzCzqhQa+eXcoVAJ2Ap
kCSuOZDHAfuZajmPk+pxkUQ5dKwN9wN/GXLpPEgtXfBD01rP0ayzuexCZjF5zKrg
Ta87nLvE6Q3iSwM7c1i3MdAtb9oWG4fRizhKLwTups/lowy6XLGmK/qZkKbwJvbO
rk78/PFT5OuGM2Y+obeZmxgY3IMJBVd1Y/Q+/MH6+rtXSRAN/XNCtXvClK9J+3PN
4967GVsEqKh0rxTHqh0bIxuO5v/EIdlLLSCyHasAVMofSqPLZIM18/wrW4+AK+g3
Yxae43T19LgSKn5hYkHfPwJN43MfQ1CQ1BAw6cIYfJRfsEb+lMa4HUERYOXHg16y
hvXs3pYhVkACz9EtkQIcIO/Dg1EP3Y6UdaPyARG52f9Tbb9g+N9OI7eOnvS5YBEB
l2fJeYwpeoGzQpzrbqQwA1IKarov8OYoJFytmQv33aPlFPPJojzDPR4gJRAgHWio
9A3pFpWRmkhJKz2WGW9CVxxYglj6r0aRKFFteEpwAayjIURzKFjr4ooOw1OJKmOB
WdaB7sKwLgEaLKXdUUf6vXmuCgn7zPt2/6CUhhmalH7+TTyrUDOMpoi2YtKbYK6Z
JDXBLk6+Ci7XwIJSmr3vPaj7aGVllLjmNKP5CFVk/qXRSE53xGv9GlfCpkk7C5Ox
P6gdOu0w20DsywhidwKBidxk9qPkFA15LXCGZ085Jl+LVHqx7N/MULjMDdKRA5j+
+F5gSPxcdA6uAzV7AKxswzpa4wnkKPZxa+9UBLY6A7M6DiR5CTRoRgjH6R4foIxz
dwul//+EeCznIwEHfl7BvVM1PTydwwyfg5Ki65kXLePex3qJuwEcHGLL9C3EI0np
nUE0sAlEuFwN0F8e9PMlkcf8ZERo/zS0t6JWnXbtydlC+T9paNGWi551U0pUfD/A
GzjmvWgkr+uj+AeY+NCr9XxH8N7OkGXPnvhezAyWbsDNdeR2Ikj8lkD+piuEv6c5
WMdW02q3YiQokJ+9NoeTPAUtrjH2GN8n9pjvNNJZshMMc5MGdF8SIFexCNNwli7z
klEjzHE9N1rmE7Cwh3dWpuk+IpNvzuBhi0/QZxUjfDrBYNUlFNfysOGx7ZejLfuY
gn4lIwFOovFD/gBfP7LW9DImIFHYm9AYQ3NVp8h9h5ZQTLneZ4idlWDK6Bpmr3sO
kWLvJguNKSQpe1ET9la8o3TKdNnNX0hws89Rdovf4V7mDi449dnS5knHs4+tlTtR
km+6BwvJ0XteVOm2HO7mmSTdTEBcc18CypvfcDawrTy8eiHNLFtZCaxW7yzecqjA
0EVUuDhx3eY/YDO1ltF2z2nFrMdTyAQmsMNW8egPZpNMo0pbyVAwelqNdRUe2ugK
Dc55WAABMYAOnYd2mkhvEbZ2NMUum+qkCelZkbrlrhl7tX0HRN4fhMbX3x+otXDC
seYYSEQF2r2l1hPiJJTOD5TWuSeWyH54n2UVgpZ2/WCZXcqViklOiyoJeojr/qqZ
r1AiLM4Zu86r2IejTlvT1CSvbOiIATsNnzHrMH0JmER7hwEXZs734LbLL9SUGh1x
uSMz2YACicH16awcK7y/gDxCmlRE68IOiJUvS8Ponn0PcT6Auxa96FPjLv2b/r7B
NWx4D+Q96CN/ojBYLk0/ANHW3C9CmjgA5nIVtHKFP1oHfw6eC08GgozcsSY2tB9v
JLlsR3dyvJBcuC64fpfAoF53QoL39+UyOBQYjZwY59pAW4NtYxvH+vBy3YrGgw4Y
T+tnIJb/Ob6A5qqAx7/T3JMSG2da3Xuemdf0fFwDdwUuWhTxZZ0ISwDUytV2xVBk
4kWz1x81shdH+3tgEJCA5yyDs6h9e/t9WlwQLHSRNupovhsA162/1Bb64B1+lOUZ
XQZWCUdhzZeoZAD0oESVTpFfbP6sy78KhlV4G+Is/TJh8+eKd20ws9jt3rv2JqKp
ydXZQeQVsRZ1Q9ZBq1lSicfXFboXtf75yBoAuwAp5C5YOml5OhTH0IWdqPENOWzZ
oga5La5H/4Xsb3AMZBwPhScpCoot/bUBl/EjOiU5KCTvjJVlzZnXLAc84sfw6ucx
0VPN/V1yl1R4hVuWsk4RvLFImGTjqLHwONtLVwoCoTlQsNzxUeBvMDLnG0DirCRX
W3CrqQu3pdTzs2TO415Cez9+pCIDvAaqGi+VzJq1Wp3n7JZdUB05H/DjZoDL6Z72
FtkDUU/g7JqVv9XsZXjf1TZT4T3evxEuNI69iiMZV5lfylLU3WEWAEPW9HQ+5GP2
t0mbdmQCbOELkG/6ApGQY3cIVZLXNf/zpttC1Q53pVbMMV6X4o+ClchHp2bb6hcF
4a3tyji925Y+9oi1FYlKFE21vwQ2Yg+ca1bxsXpTIlCswtaJfGTuYIl0qAgN2lKU
sX61CmYkX2ImxlKvhbX5K1FzUrE6i/vv4DQPahWffcIGmMyXxJpykjGchB3j/vGi
K5oflcCw/A6/vo0FyedI2PoMcFxwWMeRsuNcglj6sJa2n855F4/gHAAnLghY41zm
W23C67AD4Pokidw9N/ancxfKxTRnwpA4fA4hBNoSSK0chChOW2DduLrstNg99kSr
JTuXHtTNR62a+SAv3xk60soT2/Gw08ztNS09XKJ8qekdarbyTSU7zCjXm0NV3trY
TaPvC98Rp+AIj4ubPe+KDSi30RbBUDBpmSiwl7w8opxqi2BizpBUuL8oxn/3VzSF
Ka0Okrj+f1vdh1UuG2Qz8/WIP8IXLl9XUxT9glnSwqvPwL2mzvknZpB7oyc915y9
mcbPLR/Ye97Sp4WOYwoXe/MRBHKn1PgTqJkdA+IM9GXpIGwkrtKEXuR8DGpGnxeS
44PRnRtKAs0Cav3xd5p7390f/tWSw2o8ycX0UVc4ksADBnh70arPx+MidS6EJNAJ
+pZwTEcu0hH4VSFIcbHG3iDN7sfcddF1dG2E51B0bXjV4sVkAba6r9mhx/bwUWfl
Wj3CZ0nRetaYZRN2Y+UXxRnkgnr03nTabe95B04C4VcDW1LdYQKrGcr8dgOTWRwY
bd/SrVVFKTk6miVC2St7AsPbmI9frZ2IUZboQ3T78EbleNIrZcOUreyWxYwJHUST
viZRu8WTuHNERPq88qnZiD0HmHUaaU2lyJMYNXB/z8wS59KrJtnjcy2hjQoEbEKo
FYQxkLDR1+jKIMCNMaO9J9fkadYxQVE+yvZjXEXVcjF9cwSU/t3QjldoVpyWXKg6
GXj+SgnEc7MSp5+CN6jWw9HOrhqxVH0TA4IS1zac3TMNJKPey8++hynL4CtiSXZ5
sX6RUeFMbCVfk/MKtbozTtPIYUx+a5LObXjtWGqSGHheGfiDW/EyCXi+OvGFOmXf
mkB82pj6nYzGuHo/oAco/LlB8/T4YYtNFVcivJ3u+wZgJnQOY2r389HH6Ty53ycC
8d3FgNf05mx+X0RqyKD/GpUsmIaHkl5DgV/NVmIOGgrMzOWAp7vCCd6fVoDs+L8Q
6he6m6uOOr6s+W8XVyjko/QDGE3T2SjjPMPWnEAulZTLqng5xwv8gZ9t38jyRj84
Ai2D3T9kMFXuxIbEJCYjaONMYMbqVYDHATq7kPpTAWcYNlRXUe/Y/lcfvBiJi5mi
nJYuku+S4jkF5zA0r6xXyORTFD/edhkSB81h4luRdFOjRkTzyMpPU3CHJdG43BKU
1GJnKJLtf03jjb/RbSHOSmuNaTWb48DKbUaTcivQU47rlubboKJR5+bLZIA/PaNW
TCNe1Dx9Aik9H3mdU5+VaM0iUyf5t2gzJgnZedwMFDtH3Xb5CbQRTRXsr1h7tPO0
snwa5/9M3qUpypGx//GX1vSSDCz3fJJAYpq3WjR+V7gAK83UPi9gkLYgEgZR0OkO
0zeoSMD9GU97pTppboGHkmXzZDQzGQuGV6PxM9Kbmi16leW2JjNnudtERfoP69QU
4xgUiegY0KJwu/Ov5D+QCSxxrs6jSr+/40C21J0U5Z3b7gApHrzLteXVM+NyLZrd
uUesDygAy1LaPFR2ib/av1UCioMlwIvozGOqG6SQQ+9Sw2nQkSM3H27VPVif+SSj
4Qk1mscBLx9EG7eIjhT4lWmfb/tGKT4vlkewfUL05g2SfQxEJuHZ0UZoIYP54al2
cjN+l1SINAUED51I2UoHheLv3S5gO14ie2wy5kx3/3FHej+5YnFDEDg+mEFkdHIf
EkSHyNRnzrC5G1L/TiyCa//IVwmI9DzoMnbTFqn7OJnzeG/CxmD2mLhuReEB6wSd
njDcqhxT0Aua4vWJrhAsGLGXAI9spNbrCc4NwuOsf3JOx2BiKDgfKGEtwOYURz60
0ZAG1NBzWcs7/bYLO+5Yl89/5iryNrnVxkGqsLg5774g9uiFqE+SioOV/ABYqX50
7kjPtdlQWCc9yu2Pw/6B5D0Yf+IBMkDryzrT4QS2xkXd9zZ8lXtkyzV++N7jJ48/
Dm4b72OwpV/mq/CZ1FRM0g4+KD2G4a+dCIH9EytS/EjVNxHgoEd0Sp4krLjJDE5d
N8Ii/aKRrONlcyYT/WY9yBef7gFCwk/tsCITewGfcfdaBIZjufa/HNsqSZpMZBdz
oR7uBSJqQSKzxImQ+E79hg==
